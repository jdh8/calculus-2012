%!TEX encoding = UTF-8 Unicode
\documentclass[a4paper]{article}
\usepackage{amsmath,amsthm,amssymb}
\usepackage{CJKutf8}
\usepackage{geometry}
\usepackage{graphicx}
\usepackage[colorlinks,unicode]{hyperref}
\usepackage{pstricks-add}

\newcommand{\Left} {\mathopen{}\mathclose\bgroup\left}
\newcommand{\Right}{\aftergroup\egroup\right}
\newcommand  {\e}{\textup e}
\renewcommand{\i}{\textup i}
\newcommand  {\N}{\mathbb N}
\newcommand  {\Z}{\mathbb Z}
\newcommand  {\Q}{\mathbb Q}
\newcommand  {\R}{\mathbb R}
\renewcommand{\C}{\mathbb C}
\newcommand{\sech}  {\operatorname{sech}}
\newcommand{\csch}  {\operatorname{csch}}
\newcommand{\arsinh}{\operatorname{arsinh}}
\newcommand{\op}  {\operatorname{op}}
\newcommand{\Elem}{\operatorname{Elem}}
\newcommand{\trig}{\operatorname{trig}}
\renewcommand{\today}{\number\year~年~\number\month~月~\number\day~日}

\theoremstyle{plain}
  \newtheorem{theorem}{定理}
  \newtheorem{corollary}[theorem]{推論}
  \newtheorem{lemma}    [theorem]{引理}
\theoremstyle{definition}
  \newtheorem{definition}[theorem]{定義}
\theoremstyle{remark}
  \newtheorem{remark}{附註}

\title{指數函數}
\author{何震邦 \href{http://jdh8.org/}{\textless jdh8.org\textgreater}\\
    \href{http://creativecommons.org/licenses/by-sa/3.0/tw/deed.zh\textunderscore TW}{\includegraphics{by-sa.eps}}}

\begin{document}
\begin{CJK}{UTF8}{bsmi}
\maketitle

\begin{definition} \label{def:exp}
  對於所有 $x \in \C$,我們把指數函數定義為
  \[\exp(x) \triangleq \sum_{k=0}^{\infty} \frac{x^k}{k!}\]
  其中我們也定義 $0^0 \triangleq 1$。另外,在本文中,我們定義級數 $S$ 為
  \[S_n(x) \triangleq \sum_{k=0}^n \frac{x^k}{k!}.\]
\end{definition}

\begin{theorem}
  \textbf{比值審斂法}用來測試級數 $\displaystyle \sum_{n=0}^\infty a_n$ 是否收斂,其中每一項都是實數或複數,且當 $n$
  足夠大時 $a_n \ne 0$。

  我們設
  \[L = \lim_{n \to \infty} \left| \frac{a_{n+1}}{a_n} \right|.\]
  則
  \begin{itemize}
    \item 若 $L < 1$ 則此級數絕對收斂\footnote{\textbf{絕對收斂}即絕對值收斂至某一非負實數。}。
    \item 若 $L > 1$ 則此級數發散
    \item 若是其他狀況則沒有定論。
  \end{itemize}

  \begin{proof}
    當 $L < 1$,設 $r = \dfrac{L+1}{2}$。則 $L < 1 < r$ 且存在 $N \in \N$ 使得若 $n > N$ 則 $|a_{n+1}| < r \left| a_n
    \right|$。因此對於所有 $n > N$ 及 $k > 0$,我們都有 $|a_{n+k}| < r^k \left| a_n \right|$。所以
    \[\sum_{k=N+1}^\infty |a_k| = \sum_{k=1}^\infty |a_{N+k} < \sum_{k=1}^\infty r^k \left| a_{N+1} \right| = |a_{N+1}|
    \sum_{k=1}^\infty r^k = \frac{r \left| a_{N+1} \right|}{1-r} < \infty\]
    即此級數絕對收斂。
  \end{proof}
\end{theorem}

\begin{theorem}
  對於所有 $x \in \C$,級數 $S$ 絕對收斂。
  \begin{proof}
    \[\lim_{n \to \infty} \left| \frac{x^{n+1} / (n+1)!}{x^n / n!} \right| = \lim_{n \to \infty} \left| \frac{x}{n+1}
    \right| = 0 < 1. \qedhere\]
  \end{proof}
\end{theorem}

\begin{theorem}
  對於所有 $x,y \in \C$
  \[\exp(x+y) = \exp(x) \exp(y).\]

  \begin{proof}
    我們有
    \begin{align*}
      S_{2n}(x+y) &= \sum_{k=0}^{2n} \frac{(x+y)^k}{k!}\\
	&= \sum_{k=0}^{2n} \sum_{j=k}^{2n} \frac{x^k y^{j-k}}{k!\,(j-k)!}\\
	&= \sum_{k=0}^{2n} \sum_{j=0}^{2n-k} \frac{x^k y^j}{k!\,j!}\\
	&= \left( \sum_{k=0}^n \sum_{j=0}^n + \sum_{k=0}^{n-1} \sum_{j=n+1}^{2n-k} +  \sum_{k=n+1}^{2n} \sum_{j=0}^{2n-k}
	   \right) \frac{x^k y^j}{k!\,j!}.
    \end{align*}
    所以
    \[S_{2n}(x+y) - S_n(x)\,S_n(y) = \left( \sum_{k=0}^{n-1} \sum_{j=n+1}^{2n-k} +  \sum_{k=n+1}^{2n} \sum_{j=0}^{2n-k}
      \right) \frac{x^k y^j}{k!\,j!}.\]
    因此
    \[|S_{2n}(x+y) - S_n(x)\,S_n(y)| \le S_{n-1}(|x|) \left( S_{2n}(|y|) - S_n(|y|) \right) - S_n(|y|) \left( S_{2n}(|x|)
      - S_n(|x|) \right).\]
    因為級數 $S$ 絕對收斂,所以
    \[\lim_{n \to \infty} |S_{2n}(x+y) - S_n(x)\,S_n(y)| = 0. \qedhere\]
  \end{proof}
\end{theorem}

\begin{definition}
  我們定義數學常數 $\e$ 如下:
  \[\e \triangleq \exp(1)\]
  因此
  \[\e^x = \exp(x).\]
\end{definition}

\begin{theorem} \label{th:exp}
  \[\lim_{n \to \infty} \left( 1 + \frac x n \right)^n = \e^x.\]
  \begin{proof}
    定義序列 $T$ 使得
    \[T_n(x) \triangleq \left( 1 + \frac x n \right)^n.\]
    對於所有 $n > 2$
    \[S_n(x) - T_n(x) = \sum_{k=2}^n \frac{x^k}{k!} \left( 1 - \left( 1 - \frac1n \right) \cdots \left( 1 - \frac{k-1}{n}
      \right) \right).\]
    當 $k \ge 2$
    \[0 < 1 - \left( 1 - \frac1n \right) \cdots \left( 1 - \frac{k-1}{n} \right) \le \frac{1 + \cdots + (k-1)}{n} = \frac{k
      \left( k-1 \right)}{2n}\]
    所以
    \[|S_n(x) - T_n(x)| \le \frac{1}{2n} \sum_{k=2}^n \frac{|x|^k}{(k-2)!} = \frac{x^2 S_{n-2}(|x|)}{2n}.\]
    因此
    \[\lim_{n \to \infty} T_n(x) = \lim_{n \to \infty} S_n(x) = \e^x. \qedhere\]
  \end{proof}
\end{theorem}

\begin{theorem}
  由定義 \ref{def:exp} 得
  \[\frac{d}{dx}\,\e^x = \e^x.\]
  \begin{proof}
    \begin{align*}
      \frac{d}{dx} \sum_{k=0}^{\infty} \frac{x^k}{k!} &= \frac{d}{dx} \sum_{k=1}^{\infty} \frac{x^k}{k!}\\
	&= \sum_{k=1}^{\infty} \frac{x^{k-1}}{(k-1)!}\\
	&= \sum_{k=0}^{\infty} \frac{x^k}{k!}. \qedhere
    \end{align*}
  \end{proof}
\end{theorem}

\begin{theorem}
  由定理 \ref{th:exp} 得
  \[\frac{d}{dx}\,\e^x = \e^x.\]
  \begin{proof}
    \begin{align*}
      \frac{\e^h - 1}{h} &= \frac1h \left( \lim_{n \to \infty} \left( 1 + \frac h n \right)^n - 1 \right)\\
	&= \frac1h \left( \lim_{n \to \infty} \sum_{k=0}^n \binom n k \left( \frac h n \right)^k - 1 \right)\\
	&= \frac1h \left( \lim_{n \to \infty} \sum_{k=1}^n \binom n k \left( \frac h n \right)^k \right)\\
	&= 1 + h \left( \lim_{n \to \infty} \sum_{k=1}^n \binom n k \frac{h^{k-2}}{n^k} \right).
    \end{align*}
    因此
    \[\lim_{h \to 0} \frac{\e^h - 1}{h} = 1.\]
    由導函數的定義得
    \begin{align*}
      \frac{d}{dx}\,\e^x &= \lim_{h \to 0} \frac{\e^{x+h} - \e^x}{h}\\
	&= \lim_{h \to 0} \frac{\e^x \e^h - \e^x}{h}\\
	&= \e^x \lim_{h \to 0} \frac{\e^h - 1}{h}\\
	&= \e^x. \qedhere
    \end{align*}
  \end{proof}
\end{theorem}

\clearpage
\end{CJK}
\end{document}
