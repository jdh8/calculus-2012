%!TEX encoding = UTF-8 Unicode
\documentclass{beamer}
\usepackage{amsmath,amsthm,amssymb}
\usepackage{CJKutf8}
\usepackage{graphicx}
\usepackage{hyperref}
\usepackage{pstricks-add}

\hypersetup{colorlinks,linkcolor=,unicode}
\useoutertheme{sidebar}
\usecolortheme{rose}
\usecolortheme{seahorse}
\newcommand{\Cdot} {\!\cdot\!}
\newcommand{\Left} {\mathopen{}\mathclose\bgroup\left}
\newcommand{\Right}{\aftergroup\egroup\right}
\newcommand{\e}{\textup e}
\newcommand{\N}{\mathbb N}
\newcommand{\arsinh}{\operatorname{arsinh}}
\newcommand{\erf} {\operatorname{erf}}
\newcommand{\op}  {\operatorname{op}}
\newcommand{\Elem}{\operatorname{Elem}}
\newcommand{\trig}{\operatorname{trig}}
\renewcommand{\today}{\number\year~年~\number\month~月~\number\day~日}
\newcommand{\Negskip}{\vskip -1em plus 2pt minus 2pt}
\newcommand{\negskip}{\vskip -2em plus 3pt minus 3pt}

\theoremstyle{remark}
  \newtheorem{remark}{Remark}

\title{符號積分}
\author[何震邦]{何震邦 \href{mailto:jdh8@ms63.hinet.net}{\textless jdh8@ms63.hinet.net\textgreater}\\
    \href{http://creativecommons.org/licenses/by-sa/3.0/tw/deed.zh\textunderscore TW}{\includegraphics{by-sa.eps}}}

\begin{document}
\begin{CJK}{UTF8}{bsmi}
\maketitle

\begin{frame}{演算法}
  \begin{itemize}
    \item 1967 年,符號積分的先驅 \href{http://en.wikipedia.org/wiki/Joel\textunderscore Moses}{Joel Moses} 發表
      \href{http://www.softwarepreservation.org/projects/LISP/MIT/MIT-LCS-TR-047-corrected-ocr.pdf}{Symbolic Integration}
      這篇論文。
      \begin{itemize}
	\item 他提出的三階段演算法仍被所有主流電腦代數系統使用。
      \end{itemize}
    \item 一篇比較現代的文件是 1998 年
      \href{http://www-sop.inria.fr/cafe/Manuel.Bronstein/bronstein-eng.html}{Manuel Bronstein} 寫下的
      \href{http://www-sop.inria.fr/cafe/Manuel.Bronstein/publications/issac98.pdf}{Symbolic Integration Tutoral}。
  \end{itemize}
\end{frame}

\begin{frame}{三階段演算法}
  \begin{enumerate}
    \item 題目是簡單的題目嗎?
    \item 題目可以由變數代換解決嗎?
      \begin{itemize}
	\item 把代換後的式子丟回第一階段
	\item 或直接在這一步處理掉。
      \end{itemize}
    \item 題目可以由更一般化的方法解決嗎?
  \end{enumerate}
\end{frame}

\section{階段 1}
\begin{frame}{階段 1}
  \begin{enumerate}
    \item 若被積函數是個總和,則分別對每項進行積分,再將結果相加。
    \item 若被積函數是
      \[\left( \sum_i u_i(x) \right)^n\]
      其中 $n \in \N$,則將它展開再逐項積分。
    \item 檢查 Derivative-divides 條件,如\eqref{eq:DerivativeDivides}式。
  \end{enumerate}
\end{frame}

\subsection[逐項積分]{逐項積分與展開自然數次方}
\begin{frame}{逐項積分與展開自然數次方}
  \begin{itemize}
    \item 逐項積分:
      \[\int \left( \sin(x) + \e^x \right) dx = \int \sin(x)\,dx + \int \e^x dx.\]
    \item 展開自然數次方:
      \[\int \left( \e^x + x \right)^2 dx = \int \left( \e^{2x} + 2x\e^x + x^2 \right) dx.\]
  \end{itemize}
\end{frame}

\subsection[Deriv.\ divides]{Derivative-divides}
\begin{frame}{Derivative-divides}
  檢查原式是否符合
  \begin{equation}
    \int c \op(u(x))\,u'(x)\,dx. \label{eq:DerivativeDivides}
  \end{equation}
  其中
  \begin{columns}
    \begin{column}[t]{0.4\textwidth}
      \begin{itemize}
	\item $c$ 是常數
	\item $u(x)$ 是 $x$ 的函數
	\item $u'(x)$ 是它的導函數
      \end{itemize}
    \end{column}
    \begin{column}[t]{0.6\textwidth}
      \begin{itemize}
	\item op 是初等操作,即 op 是
	  \begin{itemize}
	    \item 三角函數
	    \item 雙曲函數
	    \item log
	  \end{itemize}
	  或 $\op(u(x))$ 具有以下的型態
	  \begin{itemize}
	    \item $1/u(x)$
	    \item $u \Left( x \Right)^a$,其中 $a \ne -1$
	    \item $b^{u(x)}$,其中 $b$ 是常數。
	  \end{itemize}
      \end{itemize}
    \end{column}
  \end{columns}
\end{frame}

\begin{frame}{$\displaystyle \int x\e^{x^2} dx$}
  \begin{solution}
    $\op(u(x)) = \e^{u(x)}$, $u(x) = x^2$, $u'(x) = 2x$, $c = \dfrac12$.
    \begin{align*}
      \int x\e^{x^2} dx &= \int \frac{\e^u}{2}\,du\\
	&= \frac{\e^u}{2}\\
	&= \frac{\e^{x^2}}{2}.
    \end{align*}
  \end{solution}
\end{frame}

\begin{frame}{$\displaystyle \int 4\cos(2x+3)\,dx$}
  \begin{solution}
    $\op = \cos$, $u(x) = 2x+3$, $u'(x) = 2$, $c = 2$.
    \begin{align*}
      \int 4\cos(2x+3)\,dx &= \int 2\cos(u)\,du\\
	&= 2\sin(u)\\
	&= 2\sin(2x+3).
    \end{align*}
  \end{solution}
\end{frame}

\begin{frame}{$\displaystyle \int \frac{\e^x}{\e^x + 1}\,dx$}
  \begin{solution}
    $\op(u(x)) = \dfrac{1}{u(x)}$, $u(x) = \e^x + 1$, $u'(x) = \e^x$, $c = 1$.
    \begin{align*}
      \int \frac{\e^x}{\e^x + 1}\,dx &= \int \frac1u\,du\\
	&= \ln \Left| u \Right|\\
	&= \ln(\e^x + 1).
    \end{align*}
  \end{solution}
\end{frame}

\begin{frame}{$\displaystyle \int \cos(x) \sin(x)\,dx$}
  \begin{solution}
    $\op(u(x)) = u(x)$, $u(x) = \cos(x)$, $u'(x) = -\sin(x)$, $c = -1$.
    \begin{align*}
      \int \cos(x) \sin(x)\,dx &= -\int u\,du\\
	&= -\frac{u^2}{2}\\
	&= -\frac{\cos \Left( x \Right)^2}{2}.
    \end{align*}
  \end{solution}
\end{frame}

\begin{frame}{$\displaystyle \int x \sqrt{x^2+1}\,dx$}
  \begin{solution}
    $\op(u(x)) = \sqrt{u(x)}$, $u(x) = x^2+1$, $u'(x) = 2x$, $c = \dfrac12$
    \begin{align*}
      \int x \sqrt{x^2+1}\,dx &= \int \frac{\sqrt u}{2}\,du\\
	&= \frac{u^{3/2}}{3}\\
	&= \frac{\left( x^2 + 1 \right)^{3/2}}{3}.
    \end{align*}
  \end{solution}
\end{frame}

\begin{frame}{$\displaystyle \int \cos \Left( \e^x \Right)^2 \sin(\e^x)\,\e^x\,dx$}
  \begin{solution}
    $\op(u(x)) = u \Left( x \Right)^2$, $u(x) = \cos(\e^x)$, $u'(x) = -\sin(\e^x)\,\e^x$, $c = -1$.
    \begin{align*}
      \int \cos \Left( \e^x \Right)^2 \sin(\e^x)\,\e^x\,dx &= -\int u^2 du\\
	&= -\frac{u^3}{3}\\
	&= -\frac{\cos \Left( \e^x \Right)^3}{3}.
    \end{align*}
  \end{solution}
\end{frame}

\section{階段 2}
\begin{frame}{初等表達式}
  當我們說某式是 $x$ 的\textbf{初等表達式},則它是由以下組成:
  \begin{enumerate}
    \item 常數
    \item $x$
    \item $x$ 的三角函數,如 $\sin(x)$、$\cos(x)$
    \item $x$ 的對數與反三角函數,如 $\ln(x)$、$\arcsin(x)$
  \end{enumerate}
  且在加法、乘法、乘方和取代下封閉,並簡記作
  \[\Elem(x).\]
  \begin{example}
    \begin{itemize}
      \item $\left( \e^x + 1 \right) \e^{2\e^x} + \e^{2x}$ 是 $\e^x$ 的初等表達式。
      \item $x\e^x$ 不是 $\e^x$ 的初等表達式,但它是 $x$ 的初等表達式。
    \end{itemize}
  \end{example}
\end{frame}

\subsection[指數替代]{指數函數的初等表達式}
\begin{frame}{方法 1:指數函數的初等表達式}
  當被積函數是 $\Elem(c^x)$,其中 $c$ 是常數,則設
  \[y = c^x.\]
  \begin{example}
    \begin{align*}
      \int \frac{\e^x}{3\e^{2x} + 2}\,dx &= \int \frac{1}{3y^2 + 2}\,dy & y = \e^x\phantom.\\
      \int \frac{\e^{x+1}}{\e^x + 1}\,dx &= \int \frac{\e}{y + 1}\,dy   & y = \e^x.
    \end{align*}
  \end{example}
\end{frame}

\begin{frame}{$\displaystyle \int \frac{\e^x}{3\e^{2x} + 2}\,dx$}
  \begin{solution}
    設 $y = \e^x$。
    \begin{align*}
      \int \frac{\e^x}{3\e^{2x} + 2}\,dx &= \int \frac{1}{3y^2 + 2}\,dy\\
	&= \frac{\arctan \Left( \frac{3y}{\sqrt6} \Right)}{\sqrt6}\\
	&= \frac{\arctan \Left( \frac{3\e^x}{\sqrt6} \Right)}{\sqrt6}.
    \end{align*}
  \end{solution}
\end{frame}

\begin{frame}{$\displaystyle \int \frac{\e^{x+1}}{\e^x + 1}\,dx$}
  \begin{solution}
    設 $y = \e^x$。
    \begin{align*}
      \int \frac{\e^{x+1}}{\e^x + 1}\,dx &= \int \frac{\e}{y + 1}\,dy\\
	&= \e \ln \Left| y+1 \Right|\\
	&= \e \ln \Left( \e^x + 1 \Right).
    \end{align*}
  \end{solution}
\end{frame}

\subsection[冪替代]{以整數次方替代}
\begin{frame}{方法 2:以整數次方替代}
  當被積函數是 $x^c \Elem(x^{k_1}, x^{k_2}, \dots)$,其中 $c$ 與 $k_i$ 是整數,則設
  \[y = x^k\]
  其中
  \[k \triangleq \gcd\{c+1, k_1, k_2, \dots\} \ne 1.\]
  \begin{example}
    \begin{align*}
      \int x^3 \sin \Left( x^2 \Right)\,dx &= \int \frac{y \sin(y)}{2}\,dy & y = x^2\phantom.\\
      \int \frac{x^7}{x^{12} + 1}\,dx &= \int \frac{y}{4y^3 + 4}\,dy       & y = x^4.
    \end{align*}
  \end{example}
\end{frame}

\begin{frame}{$\displaystyle \int x^3 \sin \Left( x^2 \Right)\,dx$}
  \begin{solution}
    設 $y = x^2$。
    \begin{align*}
      \int x^3 \sin \Left( x^2 \Right)\,dx &= \int \frac{y \sin(y)}{2}\,dy\\
	&= \frac{\sin(y) - y\cos(y)}{2}\\
	&= \frac{\sin \Left( x^2 \Right) - x^2 \cos \Left( x^2 \Right)}{2}.
    \end{align*}
  \end{solution}
\end{frame}

\begin{frame}{$\displaystyle \int \frac{x^7}{x^{12} + 1}\,dx$}
  \begin{solution}
    設 $y = x^4$。
    \begin{align*}
	 & \int \frac{x^7}{x^{12} + 1}\,dx = \int \frac{y}{4y^3 + 4}\,dy\\
      =\:& \frac{\ln \Left( y^2 - y + 1 \Right)}{24} + \frac{\arctan \Left( \frac{2y - 1}{\sqrt3} \Right)}{4\sqrt3}
	  - \frac{\ln \Left| y + 1 \Right|}{12}\\
      =\:& \frac{\ln \Left( x^8 - x^4 + 1 \Right)}{24} + \frac{\arctan \Left( \frac{2x^4 - 1}{\sqrt3} \Right)}{4\sqrt3}
	  - \frac{\ln \Left( x^4 + 1 \Right)}{12}.
    \end{align*}
  \end{solution}
\end{frame}

\subsection[根式替代]{以線性分式的 k 次方根替代}
\begin{frame}{方法 3:以線性分式的 $k$ 次方根替代}
  當被積函數是 $\Elem \Left( x, \left( \dfrac{ax + b}{cx + d} \right)^\frac{m_1}{n_1}, \left( \dfrac{ax + b}{cx + d}
  \right)^\frac{m_2}{n_2}, \dots \Right)$,其中
  \begin{itemize}
    \item 所有 $m_i$ 與 $n_i$ 互質整數,且有些 $|n_i| \ne 1$
    \item $a,b,c,d$ 為常數並 $ad - bc \ne 0$
  \end{itemize}
  則設 
  \[y = \left( \dfrac{ax + b}{cx + d} \right)^\frac1k\]
  其中 $k$ 是 $n_i$ 的最小公倍數。
\end{frame}

\begin{frame}{$\displaystyle \int \frac{1}{\sqrt x - x^{1/3}}\,dx$}
  \begin{solution}
    設 $y = x^{1/6}$,則 $dx = 6y^5 dy$。
    \begin{align*}
      \int \frac{1}{\sqrt x - x^{1/3}}\,dx &= \int \frac{6y^5}{y^3 - y^2}\,dy\\
	&= 2y^3 + 3y^2 + 6y + 6 \ln \Left| y-1 \Right|\\
	&= 2 \sqrt x + 3 x^{1/3} + 6 x^{1/6} + 6 \ln \Left| x^{1/6} - 1 \Right|.
    \end{align*}
  \end{solution}
\end{frame}

\begin{frame}{$\displaystyle \int \sqrt{\frac{x+1}{2x+3}}\,dx$,頁 I}
  \begin{columns}
    \begin{column}{0.5\textwidth}
      \begin{align*}
	y &= \sqrt{\frac{x+1}{2x+3}}\\
	2y^2 &= \frac{2x+2}{2x+3}\\
	2y^2 - 1 &= -\frac{1}{2x+3}\\
	-\frac{1}{2y^2 - 1} &= 2x+3.\\
      \end{align*}
    \end{column}
    \begin{column}{0.5\textwidth}
      \begin{align*}
	x &= -\frac{1}{2 \left( 2y^2 - 1 \right)} - \frac32\\
	\frac{dx}{dy} &= \frac{2y}{\left( 2y^2 - 1 \right)^2}.
      \end{align*}
    \end{column}
  \end{columns}
\end{frame}

\begin{frame}{$\displaystyle \int \sqrt{\frac{x+1}{2x+3}}\,dx$,頁 II}
  \begin{solution}
    設 $\displaystyle y = \sqrt{\frac{x+1}{2x+3}}$,則 $dx = \dfrac{2y}{\left( 2y^2 - 1 \right)^2}\,dy$。
    \begin{align*}
      \int \sqrt{\frac{x+1}{2x+3}}\,dx &= \int \frac{2y^2}{\left( 2y^2 - 1 \right)^2}\,dy  \qquad \\
	&= \frac{\ln \Left| \frac{2y - \sqrt2}{2y + \sqrt2} \Right|}{2^{5/2}} - \frac{y}{4y^2 - 2}\\
	&= \frac{\ln \Left| \frac{2\sqrt{\frac{x+1}{2x+3}} - \sqrt2}{2\sqrt{\frac{x+1}{2x+3}} + \sqrt2} \Right|}{2^{5/2}}
	  + \frac{(2x+3) \sqrt{\frac{x+1}{2x+3}}}{2}.
    \end{align*}
  \end{solution}
\end{frame}

\subsection[切比雪夫積分]{二項式--切比雪夫積分}
\begin{frame}{方法 4:二項式--切比雪夫積分}
  \[\int x^p \left( c_1 x^q + c_0 \right)^r dx\]
  其中 $p,q,r$ 是有理數。設 $y = x^q$ 則 $dy = qx^{q-1} dx$
  \[\int \frac{y^{\frac{p+1}{q}-1}}{q} \left( c_1 y + c_0 \right)^r dy.\]
  \begin{remark}
    當 $r \in \N$ 的時候不要衝動!直接展開不是比較舒服嗎?
  \end{remark}
\end{frame}

\begin{frame}{切比雪夫積分}
  \[\int x^{r_1} \left( c_1 x + c_0 \right)^{r_2} dx\]
  其中 $r_1, r_2$ 是有理數,$r_1$ 的分母是 $n_1$,$r_2$ 的分母是 $n_2$。
  \begin{columns}
    \begin{column}[t]{0.5\textwidth}
      \begin{itemize}
	\item 當 $r_1$ 是正整數
	  \[z = c_1 x + c_0.\]
	\item 當 $r_1$ 是負整數
	  \[z = \left( c_1 x + c_0 \right)^{1/n_2}.\]
      \end{itemize}
    \end{column}
    \begin{column}[t]{0.5\textwidth}
      \begin{itemize}
	\item 當 $r_2$ 是負整數
	  \[z = x^{1/n_1}.\]
	\item 當 $r_1 + r_2$ 是整數
	  \[z = \frac{\left( c_1 x + c_0 \right)^{1/n_1}}{x^{1/n_1}}.\]
      \end{itemize}
    \end{column}
  \end{columns}
\end{frame}

\begin{frame}{$\displaystyle \int x^2 \left( c_1 x + c_0 \right)^{3/5} dx$}
  \begin{solution}
    設 $z = c_1 x + c_0$ 則 $dx = \dfrac{dz}{c_1}$。
    \begin{align*}
	 & \int x^2 \left( c_1 x + c_0 \right)^{3/5} dx\\
      =\:& \int \left( \frac{z^{13/5}}{c_1^3} - \frac{2 c_0 z^{8/5}}{c_1^3}
	  + \frac{c_0^2 z^{3/5}}{c_1^3} \right) dz\\
      =\:& \frac{5z^{18/5}}{18} - \frac{10 c_0 z^{13/5}}{13 c_1^3} + \frac{5 c_0^2 z^{8/5}}{8 c_1^3}\\
      =\:& \frac{5 \left( c_1 x + c_0 \right)^{18/5}}{18 c_1^3}
	  - \frac{10 c_0 \left( c_1 x + c_0 \right)^{13/5}}{13 c_1^3}
	  + \frac{5 c_0^2 \left( c_1 x + c_0 \right)^{8/5}}{8 c_1^3}.
    \end{align*}
  \end{solution}
\end{frame}

\begin{frame}{$\displaystyle \int \frac{1}{\sqrt{\sqrt{x+1}+1}}\,dx$}
  \begin{solution}
    設 $y = \sqrt{x+1}$,則 $dx = 2y\,dy$。
    \[\int \frac{1}{\sqrt{\sqrt{x+1}+1}}\,dx = \int \frac{2y}{\sqrt{y+1}}\,dy.\]
    設 $z = y+1$。
    \begin{align*}
      \int \frac{2y}{\sqrt{y+1}}\,dy &= \int \left( 2 \sqrt z - \frac{2}{\sqrt z} \right) dz\\
	&= \frac{4z^{3/2}}{3} - 4 \sqrt z\\
	&= \frac{4 \left( \sqrt{x+1}+1 \right)^{3/2}}{3} - 4 \sqrt{\sqrt{x+1}+1}.
    \end{align*}
  \end{solution}
\end{frame}

\begin{frame}{更多例子}
  \begin{example}
    \[\int \frac{\left( cx-8 \right)^{2/3}}{x}\,dx = \int \frac{3z^4}{z^3 + 8}\,dz\]
    其中 $z = \left( cx-8 \right)^{1/3}$,所以 $dx = \dfrac{3z^2}{c}\,dz$。
  \end{example}
  \begin{example}
    \[\int \frac{x^{3/4}}{\left( x+c \right)^3}\,dx = \int \frac{4z^6}{\left( z^4 + c \right)^3}\,dz\]
    其中 $z = x^{1/4}$,所以 $dx = 4z^3$。
  \end{example}
\end{frame}

\begin{frame}{$\displaystyle \int \sqrt x \left( x+1 \right)^{5/2} dx$}
  \begin{example}
    設 $\displaystyle z = \frac{\sqrt{x+1}}{\sqrt x}$,所以 $dx = -\dfrac{2z}{\left( z^2 - 1 \right)^2}\,dz$。
    \[\int \sqrt x \left( x+1 \right)^{5/2} dx = \int -\frac{2z^6}{\left( z^2 - 1 \right)^5}\,dz.\]
  \end{example}
\end{frame}

\subsection{反三角替代}
\begin{frame}{方法 5:反三角替代}
  \centerline{詳見《無理函數的積分》}
\end{frame}

\subsection[三角函數]{三角函數的初等表達式}
\begin{frame}{方法 6:三角函數的積分}
  \begin{itemize}
    \item 若被積函數中有 $\trig_1(ax+b) \trig_2(cx+d)$,其中 $\trig_1$ 與 $\trig_2$ 是 sin 或 cos,則進行積化和差。
    \item 若被積函數是 $\trig_1(nx) \trig_2(y)$,其中 $\trig_1$ 是 sin 或 cos,$n$ 是整數,且 $\trig_2$ 不是 sin 或
      cos,則將 $\trig_1(nx)$ 以倍角公式展開。
    \item 若被積函數中所有三角函數的參數都是 $ax+b$,則設新變數 $y = ax+b$。
  \end{itemize}
\end{frame}

\begin{frame}{$\displaystyle \int \cos(mx) \cos(nx)\,dx$}
  \begin{solution}
    \begin{align*}
	 & \int \cos(mx) \cos(nx)\,dx\\
      =\:& \int \frac{\cos((m+n)x)}{2}\,dx + \int \frac{\cos((m-n)x)}{2}\,dx\\
      =\:& \frac{\sin((m+n)x)}{2 \left( m+n \right)} + \frac{\sin((m-n)x)}{2 \left( m-n \right)}.
    \end{align*}
  \end{solution}
\end{frame}

\begin{frame}{$\displaystyle \int \sin(2x) \tan(x)\,dx$}
  \begin{solution}
    \begin{align*}
      \int \sin(2x) \tan(x)\,dx &= \int 2 \cos(x) \sin(x) \tan(x)\,dx\\
	&= \int 2 \sin \Left( x \Right)^2 dx\\
	&= \int (1 - \cos(2x))\,dx\\
	&= x - \frac{\sin(2x)}{2}.
    \end{align*}
  \end{solution}
\end{frame}

\begin{frame}{三角函數的積分,單一參數}
  把所有參數化為單一變數 $x$ 後
  \begin{itemize}
    \item 將所有三角函數化為 sin 和 cos
      \begin{itemize}
	\item $\cos \Left( x \Right)^{2m} \sin \Left( x \Right)^{2n}$,其中 $\{m,n\} \in \N_0$。
	\item $\cos \Left( x \Right)^{2n+1} \Elem \Left( \cos \Left( x \Right)^2, \sin(x) \Right)$ 設 $y = \sin(x)$。
	\item $\sin \Left( x \Right)^{2n+1} \Elem \Left( \sin \Left( x \Right)^2, \cos(x) \Right)$ 設 $y = \cos(x)$。
      \end{itemize}
    \item 若行不通,將所有三角函數化為 tan 和 sec
      \begin{itemize}
	\item $\sec \Left( x \Right)^{2n} \Elem \Left( \tan(x), \sec \Left( x \Right)^2 \Right)$ 設 $y = \tan(x)$。
      \end{itemize}
    \item 最後只好使用萬能的
      \[y = \tan \Left( \frac x 2 \Right) = \frac{\sin(x)}{\cos(x) + 1}.\]
  \end{itemize}
\end{frame}

\begin{frame}{$\displaystyle \int \cos \Left( x \Right)^{2m} \sin \Left( x \Right)^{2n} dx$}
  \begin{columns}
    \begin{column}{0.4\textwidth}
      \begin{theorem}
	\[\cos(x) \sin(x) = \frac{\sin(2x)}{2}.\]
	\begin{align*}
	  \cos \Left( x \Right)^2 &= \frac{1 + \cos(2x)}{2}\\
	  \sin \Left( x \Right)^2 &= \frac{1 - \cos(2x)}{2}.
	\end{align*}
      \end{theorem}
    \end{column}
    \begin{column}{0.6\textwidth}
      當 $m \ge n$,化為
      \[\left( \left( \frac{1 + \cos(2x)}{2} \right)^{m-n} \frac{\sin(2x)}{2} \right)^{2n}.\]
      當 $m < n$,化為
      \[\left( \left( \frac{1 - \cos(2x)}{2} \right)^{n-m} \frac{\sin(2x)}{2} \right)^{2m}.\]
    \end{column}
  \end{columns}
\end{frame}

\begin{frame}[allowframebreaks]{$\displaystyle \int_{\frac\pi4}^{3\pi} \cos \Left( x \Right)^6 \sin \Left( x \Right)^4
    dx$,頁}
  \negskip
  \begin{align*}
       & \int \cos \Left( x \Right)^6 \sin \Left( x \Right)^4 dx\\
    =\:& \int \frac{(\cos(2x) + 1) \sin \Left( 2x \Right)^4}{32}\,dx\\
    =\:& \int \left( \frac{\cos(2x) \sin \Left( 2x \Right)^4}{32} + \frac{\sin \Left( 2x \Right)^4}{32} \right) dx\\
    =\:& \frac{\sin \Left( 2x \Right)^5}{320} + \int \frac{\cos \Left( 4x \Right)^2 - 2\cos(4x) + 1}{128}\,dx\\
    =\:& -\frac{\sin(4x)}{256} + \frac{\sin \Left( 2x \Right)^5}{320} + \int \frac{\cos(8x) + 3}{256}\,dx\\
    =\:& \frac{\sin(8x)}{2048} - \frac{\sin(4x)}{256} + \frac{\sin \Left( 2x \Right)^5}{320} + \frac{3x}{256}.
  \end{align*}
  \begin{align*}
    \int_{\frac\pi4}^{3\pi} \cos \Left( x \Right)^6 \sin \Left( x \Right)^4 dx
      &= \frac{9\pi}{256} - \frac{15\pi + 16}{5120}\\
      &\approx 0.09811773200045232.
  \end{align*}
\end{frame}

\begin{frame}{$\displaystyle \int \sec \Left( x \Right)^3 dx$}
  \begin{solution}
    設 $y = \sin(x)$。
    \begin{align*}
	 & \int \sec \Left( x \Right)^3 dx\\
      =\:& \int \frac{1}{\left( 1 - y^2 \right)^2}\,dy\\
      =\:& \frac{\ln \Left| y + 1 \Right|}{4} - \frac{\ln \Left| y - 1 \Right|}{4} - \frac{y}{2y^2 - 2}\\
      =\:& \frac{\ln \Left| \sin(x) + 1 \Right|}{4} - \frac{\ln \Left| \sin(x) - 1 \Right|}{4} 
	  - \frac{\sin(x)}{2 \sin \Left( x \Right)^2 - 2}.
    \end{align*}
  \end{solution}
\end{frame}

\begin{frame}{$\displaystyle \int \frac{\sec \Left( x \Right)^2}{3\tan(x) + \sec \Left( x \Right)^2 + 1}\,dx$}
  \begin{solution}
    設 $y = \tan(x)$。
    \begin{align*}
	 & \int \frac{\sec \Left( x \Right)^2}{3\tan(x) + \sec \Left( x \Right)^2 + 1}\,dx\\
      =\:& \int \frac{1}{y^2 + 3y + 2}\,dy\\
      =\:& \int \left( \frac{1}{y+1} - \frac{1}{y+2} \right) dy\\
      =\:& \ln \Left| y+1 \Right| - \ln \Left| y+2 \Right|\\
      =\:& \ln \Left| \tan(x) + 1 \Right| - \ln \Left| \tan(x) + 2 \Right|.
    \end{align*}
  \end{solution}
\end{frame}

\begin{frame}{$\displaystyle \int \frac{1}{\cos(x) + 1}\,dx$}
  \begin{solution}
    設 $y = \dfrac{\sin(x)}{\cos(x) + 1}$,則 $dx = \dfrac{2}{y^2 + 1}\,dy$。
    \begin{align*}
      \int \frac{1}{\cos(x) + 1} &= \int \frac{2}{\left( y^2 + 1 \right) \left( \frac{1-y^2}{1+y^2} + 1 \right)}\,dy\\
	&= \int dy\\
	&= \frac{\sin(x)}{\cos(x) + 1}.
    \end{align*}
  \end{solution}
\end{frame}

\subsection[有理函數乘以指數]{有理函數乘以指數函數}
\begin{frame}{方法 7:有理函數乘以指數函數}
  \[\int \frac{N \e^P}{D}\]
  其中 $D$, $N$, $P$ 是多項式且 $D$ 與 $N$ 互質。根據\href{http://en.wikipedia.org/wiki/Liouville\%27s\textunderscore%
  theorem\textunderscore \%28differential\textunderscore algebra\%29}{劉維爾定理},若此積分能以有限項表達,則必存在有理函數
  $A$ 使得
  \[A \e^P = \int \frac{N \e^P}{D}.\]
\end{frame}

\begin{frame}{標準作業程序 I}
  設 $N$ 的領導項為 $c_1 x^m$,其中 $m \in \N_0$。即
  \[N = c_1 x^{m_1} + S_1\]
  其中 $c_1 \ne 0$ 且多項式 $S_1$ 滿足 $\deg(S_1) < m$。則設有理函數 $A_1$ 與 $B_1$ 使得
  \begin{align}
    A_1 &= \frac{c_1 x^{m_1}}{DP'} \nonumber \\
    \left( A_1 + B_1 \right) \e^P &= \int \frac{N \e^P}{D}. \label{eq:A+B=N/D}
  \end{align}
\end{frame}

\begin{frame}{標準作業程序 II}
  根據\eqref{eq:A+B=N/D}式,我們有
  \[B_1 P' + B_1' = \frac N D - A_1 P' - A_1' = \frac{Q_1 + R_1}{D}.\]
  其中 $Q_1$ 為多項式,$R_1$ 為真分式。若 $Q_{k-1} \ne 0$ 則設 $Q_{k-1}$ 的領導項為 $c_k x^{m_k}$,並設有理函數
  $A_k$ 與 $B_k$ 使得
  \begin{align*}
    A_k &= \frac{c_k x^{m_k}}{DP'}\\
    \left( A_k + B_k \right) \e^P &= \int \frac{N_k \e^P}{D}
  \end{align*}
  其中 $N_k = Q_{k-1} + R_{k-1}$。重複以上算法到 $Q_n = 0$ 為止。
\end{frame}

\begin{frame}{標準作業程序 III}
  設 $\displaystyle A = \sum_{k=1}^n A_k$。
  \begin{align*}
    \frac N D &= P'A + A' + \frac{R_n}{D}\\
    \frac{N \e^P}{D} &= P'A \e^P + A' \e^P + \frac{R_n \e^P}{D}\\
      &= \left( A \e^P \right)' + \frac{R_n \e^P}{D}\\
    \int \frac{N \e^P}{D} &= A \e^P + \int \frac{R_n \e^P}{D}
  \end{align*}
  其中 $\displaystyle \int \frac{R_n \e^P}{D}$ 必無法以有限項表達。
\end{frame}

\begin{frame}{$\displaystyle \int \frac{x \e^x}{\left( x+1 \right)^2}\,dx$,第 1 步}
  已知 $P' = 1$ 且 $D = \left( x+1 \right)^2$。
  \begin{align*}
    A_1  &= \frac x D\\
    A_1' &= \frac{x-1}{\left( x+1 \right)^3} = \frac{2}{\left( x+1 \right)^3} - \frac{1}{\left( x+1 \right)^2}\\
    B_1 P' + B_1' &= \frac{1 + R_1}{D}\\
    R_1  &= -\frac{2}{x+1}.
  \end{align*}
\end{frame}

\begin{frame}{$\displaystyle \int \frac{x \e^x}{\left( x+1 \right)^2}\,dx$,第 2 步}
  \begin{align*}
    A_2  &= \frac1D\\
    A_2' &= -\frac{2}{\left( x+1 \right)^3} = \frac{R_1}{D}\\
    B_2 P' + B_2' &= 0.
  \end{align*}
  因此
  \begin{align*}
    A = A_1 + A_2 &= \frac{1}{x+1}\\
    \int \frac{x \e^x}{\left( x+1 \right)^2}\,dx &= \frac{\e^x}{x+1}.
  \end{align*}
\end{frame}

\begin{frame}{$\displaystyle \int \frac{\left( 2x^6 + 5x^4 + x^3 + 4x^2 + 1 \right) \e^{x^2}}{\left( x^2 + 1 \right)^2}
    \,dx$,第 1 步}
  已知 $P' = 2x$ 且 $D = \left( x^2 + 1 \right)^2$。
  \begin{align*}
    A_1  &= \frac{2x^6}{DP'} = \frac{x^5}{D}\\
    A_1' &= \frac{5x^4}{\left( x^2 + 1 \right)^2} - \frac{4x^6}{\left( x^2 + 1 \right)^3}\\
    B_1 P' + B_1' &= \frac{4x^4 + x^3 + 5 + R_1}{D}\\
    R_1  &= -\frac{4}{x^2 + 1}.
  \end{align*}
\end{frame}

\begin{frame}{$\displaystyle \int \frac{\left( 2x^6 + 5x^4 + x^3 + 4x^2 + 1 \right) \e^{x^2}}{\left( x^2 + 1 \right)^2}
    \,dx$,第 2 步}
  \begin{align*}
    A_2  &= \frac{4x^4}{DP'} = \frac{2x^3}{D}\\
    A_2' &= \frac{6x^2}{\left( x^2 + 1 \right)^2} - \frac{8x^4}{\left( x^2 + 1 \right)^3}\\
    B_2 P' + B_2' &= \frac{x^3 + 2x^2 - 3 + R_2}{D}\\
    R_2  &= \frac{4}{x^2 + 1}.
  \end{align*}
\end{frame}

\begin{frame}{$\displaystyle \int \frac{\left( 2x^6 + 5x^4 + x^3 + 4x^2 + 1 \right) \e^{x^2}}{\left( x^2 + 1 \right)^2}
    \,dx$,第 3 步}
  \begin{align*}
    A_3  &= \frac{x^3}{DP'} = \frac{x^2}{2D}\\
    A_3' &= \frac{x}{\left( x^2 + 1 \right)^2} - \frac{2x^3}{\left( x^2 + 1 \right)^3}\\
    B_3 P' + B_3' &= \frac{2x^2 + x - 3 + R_3}{D}\\
    R_3  &= \frac{4 - 2x}{x^2 + 1}.
  \end{align*}
\end{frame}

\begin{frame}{$\displaystyle \int \frac{\left( 2x^6 + 5x^4 + x^3 + 4x^2 + 1 \right) \e^{x^2}}{\left( x^2 + 1 \right)^2}
    \,dx$,第 4 步}
  \begin{align*}
    A_4  &= \frac{2x^2}{DP'} = \frac x D\\
    A_4' &= \frac{1}{\left( x^2 + 1 \right)^2} + \frac{4x^2}{\left( x^2 + 1 \right)^3}\\
    B_4 P' + B_4' &= \frac{x + R_4}{D}\\
    R_4  &= -\frac{2x}{x^2 + 1}.
  \end{align*}
\end{frame}

\begin{frame}{$\displaystyle \int \frac{\left( 2x^6 + 5x^4 + x^3 + 4x^2 + 1 \right) \e^{x^2}}{\left( x^2 + 1 \right)^2}
    \,dx$,第 5 步}
  \begin{align*}
    A_5  &= \frac{x}{DP'} = \frac{1}{2D}\\
    A_5' &= -\frac{2x}{\left( x^2 + 1 \right)^3}\\
    B_5 P' + B_5' &= 0.
  \end{align*}
  因此
  \begin{align*}
    A &= \sum_{k=1}^5 A_k = \frac{1}{x+1}\\
    \int \frac{N \e^P}{D} &= \frac{2x^5 + 4x^3 + x^2 + 2x + 1}{2D} = \frac{2x^3 + 2x + 1}{2x^2 + 2}.
  \end{align*}
\end{frame}

\begin{frame}{誤差函數}
  \begin{definition}
    \textbf{誤差函數}是一個非初等函數。它的定義如下:
    \[\erf(x) = \frac{2}{\sqrt\pi} \int_0^x \e^{-t^2}\,dt.\]
  \end{definition}
  \begin{center}
    \begin{pspicture}(-3,-1)(3,1)
      \psaxes(0,0)(-3,-1)(3,1)
      \psplot[plotstyle=curve]{-3}{3}{0 x /t {t t mul neg EXP} 0.0009765625 SIMPSON 1.128379167095513 mul}
    \end{pspicture}
  \end{center}
\end{frame}

\begin{frame}{常態分佈}
  \begin{definition}
    若隨機變數 $X$ 服從期望值為 $\mu$、標準差為 $\sigma$ 的機率分佈,記作
    \[X \sim N \Left( \mu, \sigma^2 \Right)\]
    則其機率密度函數為
    \[f(x) = \frac{\exp \Left( -\frac{\left( x - \mu \right)^2}{2\sigma^2} \Right)}{\sqrt{2\pi} \sigma}.\]
  \end{definition}
\end{frame}

\begin{frame}{常態分佈的累積分佈函數}
  \begin{theorem}
    常態分佈的累積分佈函數為
    \[F(x) = \frac12 + \frac12 \erf \Left( \frac{x-\mu}{\sqrt2 \sigma} \Right).\]
  \end{theorem}
  \begin{proof}
    \negskip
    \begin{align*}
      F(x) &= \frac{1}{\sqrt{2\pi} \sigma} \int_{-\infty}^x \exp
	  \Left( -\frac{\left( t - \mu \right)^2}{2\sigma^2} \Right)\,dt\\
	&= \frac12 + \frac12 \erf \Left( \frac{x-\mu}{\sqrt2 \sigma} \Right).\qedhere
    \end{align*}
  \end{proof}
\end{frame}

\begin{frame}{$\displaystyle \int_0^\infty \e^{-x^2} dx = \frac{\sqrt\pi}{2}$}
  \begin{proof}
    設 $\displaystyle J = \int_0^\infty \e^{-x^2} dx$,顯然 $J > 0$。
    \[J^2 = \left( \int_0^\infty \e^{-x^2} dx \right)^2 = \int_0^\infty \int_0^\infty \e^{-x^2-y^2} dx\,dy.\]
    設 $r$ 與 $\theta$ 使得 $x = r\cos\theta$ 且 $y = r\sin\theta$,即極座標。積分範圍為第一象限。積分單位由 $dx\,dy$ 轉為
    $r\,dr\,d\theta$。
    \[J^2 = \int_0^{\frac\pi2} \int_0^\infty r \e^{-r^2} dr\,d\theta
	= \int_0^{\frac\pi2} \left[ -\frac{\e^{-r^2}}{2} \right]_0^\infty d\theta
	= \left[ \frac\theta2 \right]_0^{\frac\pi2}
	= \frac\pi4\]
    \[J = \frac{\sqrt\pi}{2}.\qedhere\]
  \end{proof}
\end{frame}

\subsection{有理函數}
\begin{frame}{方法 8:有理函數的積分}
  \centerline{詳見《有理函數的積分》}
\end{frame}

\subsection[有理函數乘以反函數]{有理函數乘以對數或反三角函數}
\begin{frame}{方法 9:有理函數乘以對數或反三角函數}
  \[\int f'(x) \op(g(x))\,dx\]
  其中 $f$, $f'$ 與 $g$ 均為有理函數,op 為對數或反三角函數。則設 $u = f(x)$ 與 $v = g(x)$ 再進行分部積分,即
  \[\int \op(v)\,du = u \op(v) - \int u \op'(v)\,dv.\]
\end{frame}

\begin{frame}{$\displaystyle \int f'(x) \ln \Left| g(x) \Right|\,dx$}
  \begin{solution}
    設 $u = f(x)$ 與 $v = g(x)$。
    \begin{align*}
      \int \ln \Left| v \Right|\,du &= u \ln \Left| v \Right| - \int \frac u v \,dv\\
	&= f(x) \ln \Left| g(x) \Right| - \int \frac{f(x)\,g'(x)}{g(x)}\,dx.
    \end{align*}
  \end{solution}
\end{frame}

\begin{frame}{$\displaystyle \int f'(x) \arctan(g(x))\,dx$}
  \begin{solution}
    設 $u = f(x)$ 與 $v = g(x)$。
    \begin{align*}
      \int \arctan(v)\,du &= u \arctan(v) - \int \frac{u}{v^2 + 1}\,dv\\
	&= f(x) \arctan(g(x)) - \int \frac{f(x)\,g'(x)}{g \Left( x \Right)^2 + 1}\,dx.
    \end{align*}
  \end{solution}
\end{frame}

\begin{frame}{$\displaystyle \int f'(x) \arcsin(g(x))\,dx$}
  \begin{solution}
    設 $u = f(x)$ 與 $v = g(x)$。
    \begin{align*}
      \int \arcsin(v)\,du &= u \arcsin(v) - \int \frac{u}{\sqrt{1 - v^2}}\,dv\\
	&= f(x) \arcsin(g(x)) - \int \frac{f(x)\,g'(x)}{\sqrt{1 - g \Left( x \Right)^2}}\,dx.
    \end{align*}
  \end{solution}
\end{frame}

\begin{frame}{$\displaystyle \int f'(x) \arsinh(g(x))\,dx$}
  \begin{solution}
    設 $u = f(x)$ 與 $v = g(x)$。
    \begin{align*}
      \int \arsinh(v)\,du &= u \arsinh(v) - \int \frac{u}{\sqrt{v^2 + 1}}\,dv\\
	&= f(x) \arsinh(g(x)) - \int \frac{f(x)\,g'(x)}{\sqrt{g \Left( x \Right)^2 + 1}}\,dx.
    \end{align*}
  \end{solution}
\end{frame}

\begin{frame}{$\displaystyle \int 2x \ln \Left| x \Right|\,dx$}
  \begin{solution}
    \begin{align*}
      \int 2x \ln \Left| x \Right|\,dx &= x^2 \ln \Left| x \Right| - \int x\,dx\\
	&= x^2 \ln \Left| x \Right| - \frac{x^2}{2}.
    \end{align*}
  \end{solution}
\end{frame}

\begin{frame}{$\displaystyle \int x^2 \arcsin(x)\,dx$}
  \begin{solution}
    \begin{align*}
      \int x^2 \arcsin(x)\,dx &= \frac{x^3 \arcsin(x)}{3} - \int \frac{x^3}{3 \sqrt{1 - x^2}}\,dx\\
	&= \frac{x^3 \arcsin(x)}{3} + \frac{\sqrt{1 - x^2} \left( x^2 + 2 \right)}{9}.
    \end{align*}
  \end{solution}
\end{frame}

\begin{frame}{$\displaystyle \int \frac{\ln \Left| x^2 + 2x \Right|}{x^2 + 2x + 1}\,dx$}
  \begin{solution}
    \begin{align*}
      \int \frac{1}{x^2 + 2x + 1}\,dx &= -\frac{1}{x+1}\\
      \int \frac{\ln \Left| x^2 + 2x \Right|}{x^2 + 2x + 1}\,dx
	&= -\frac{\ln \Left| x^2 + 2x \Right|}{x+1} + \int \frac{2x + 2}{(x+1) \left( x^2 + 2x \right)}\,dx\\
	&= -\frac{\ln \Left| x^2 + 2x \Right|}{x+1} + \int \frac{2}{x^2 + 2x}\,dx\\
	&= -\frac{\ln \Left| x^2 + 2x \Right|}{x+1} - \ln \Left| x+2 \Right| + \ln \Left| x \Right|.
    \end{align*}
  \end{solution}
\end{frame}

\subsection[對數替代]{有理函數乘以對數函數的初等表達式}
\begin{frame}{方法 10:有理函數乘以對數函數的初等表達式}
  當被積函數是有理函數 $f$ 乘以 $\Elem(\log_c(ax + b))$,則設
  \[y = \log_c(ax + b).\]
  此時
  \[\int f(x) \Elem(\log_c(ax + b))\,dx = \int \frac{\ln(c)\,c^y f(z) \Elem(y)}{b}\,dy\]
  其中 $z = \dfrac{c^y - a}{b}$。通常這樣能把問題丟給方法 7 或階段 3。
\end{frame}

\subsection{展開被積函數}
\begin{frame}{方法 11:展開被積函數}
  當以上方法都無效時,不妨試試把被積函數乘開。
  \begin{example}
    \begin{align*}
      x \left( \sin(x) + \cos(x) \right) &= x \sin(x) + x \cos(x)\\
      \frac{\e^x + x}{\e^x} &= x \e^{-x} + 1\\
      x \left( \e^x + 1 \right)^2 &= x \e^{2x} + 2x \e^x + x.
    \end{align*}
  \end{example}
\end{frame}

\section{階段 3}
\subsection{分部積分}
\begin{frame}{採用分部積分的時機}
  \begin{itemize}
    \item 多項式乘以指數函數
    \item 多項式乘以 sin 或 cos
    \item 指數函數乘以 sin 或 cos
    \item 多項式乘以指數函數乘以 sin 或 cos。
  \end{itemize}
\end{frame}

\begin{frame}{表格解法}
  \begin{theorem}
    \begin{itemize}
      \item 初始時,取易微分的部份為 $u$,剩下的部份為 $dv$,並找出 $v = \int dv$。
      \item 取下一個 $u$ 為 $u'$,$v$ 為 $-\int v\,dx$,直到 $u$ 為常數。
      \item 答案為 $\sum uv$。
    \end{itemize}
  \end{theorem}
  \begin{remark}
    \[\int u'(x)\,v(x)\,dx = \int v\,du.\]
  \end{remark}
\end{frame}

\begin{frame}{多項式乘以指數函數}
  \begin{itemize}
    \item 遇 cosh 與 sinh 都化為指數函數。
    \item 若指數函數的參數為 $ax+b$,則將 $b$ 提出為常數。
    \item 分部積分時,取多項式為 $u$。
  \end{itemize}
\end{frame}

\begin{frame}{$\displaystyle \int x^3 \e^{ax}\,dx$}
  \begin{solution}
    \begin{center}
      \begin{tabular}{cc}
	$u$  & $v$\\
	\hline
	$x^3$ & $\phantom+\e^{ax}/a^{\phantom1}$\\
	$3x^2$& $        -\e^{ax}/a^2$\\
	$6x$  & $\phantom+\e^{ax}/a^3$\\
	6     & $        -\e^{ax}/a^4$
      \end{tabular}
    \end{center}
    \[\int x^3 \e^{ax} dx = \frac{\left( a^3 x^3 - 3a^2 x^2 + 6ax - 6 \right) \e^{ax}}{a^4}.\]
  \end{solution}
\end{frame}

\begin{frame}{多項式乘以 sin 或 cos}
  \begin{itemize}
    \item 遇三角函數的乘積,一律採取積化和差,否則三角函數可能會越積越大串。
    \item 若三角函數的參數為 $ax+b$,則設新變數 $y = ax+b$。
    \item 分部積分時,取多項式為 $u$。
  \end{itemize}
\end{frame}

\begin{frame}{$\displaystyle \int x^2 \cos(x)\,dx$}
  \begin{solution}
    設 $u = x^2$ 與 $v = \sin(x)$。
    \[\int u\,dv = uv - \int v\,du = x^2 \sin(x) - \int 2x \sin(x)\,dx.\]
    設 $u = 2x$ 與 $v = \cos(x)$。
    \[\int u\,dv = uv - \int v\,du = 2x \cos(x) - \int 2\cos(x)\,dx.\]
    所以
    \[\int x^2 \cos(x)\,dx = \left( x^2 - 2 \right) \sin(x) + 2x \cos(x).\]
  \end{solution}
\end{frame}

\begin{frame}{$\displaystyle \int x^2 \cos(x)\,dx$,表格解法}
  \begin{solution}
    \begin{center}
      \begin{tabular}{cc}
	$u$  & $v$\\
	\hline
	$x^2$& $\phantom+\sin(x)$\\
	$2x$ & $\phantom+\cos(x)$\\
	2    & $        -\sin(x)$
      \end{tabular}
    \end{center}
    \[\int x^2 \cos(x)\,dx = \left( x^2 - 2 \right) \sin(x) + 2x \cos(x).\]
  \end{solution}
\end{frame}

\begin{frame}{$\displaystyle \int x^2 \cos(3x)\,dx$}
  \begin{solution}
    設 $y = 3x$。
    \begin{align*}
      \int x^2 \cos(3x)\,dx &= \int \frac{y^2 \cos(y)}{27}\,dy\\
	&= \frac{\left( y^2 - 2 \right) \sin(y) + 2y \cos(y)}{27}\\
	&= \frac{\left( 9x^2 - 2 \right) \sin(3x) + 6x \cos(3x)}{27}.
    \end{align*}
  \end{solution}
\end{frame}

\begin{frame}{$\displaystyle \int x^2 \cos(x) \sin \Left( x \Right)^2 dx$}
  \begin{solution}
    \begin{align*}
	 & \int x^2 \cos(x) \sin \Left( x \Right)^2 dx\\
      =\:& \int \frac{x^2 \cos(x) - x^2 \cos(3x)}{4}\,dx\\
      =\:& \frac{\left( x^2 - 2 \right) \sin(x) + 2x \cos(x)}{4}\\
      \phantom=\:& -\frac{\left( 9x^2 - 2 \right) \sin(3x) + 6x \cos(3x)}{108}.
    \end{align*}
  \end{solution}
\end{frame}

\begin{frame}[allowframebreaks]{$\displaystyle \int_{\frac\pi6}^{\frac\pi4} x^2 \sin(3x)\,dx$,頁}
  設 $y = 3x$。
  \[\int x^2 \sin(3x)\,dx = \int \frac{y^2 \sin(y)}{27}\,dy.\]
  \begin{center}
    \begin{tabular}{cc}
      $u$  & $v$\\
      \hline
      $y^2$& $        -\cos(y)$\\
      $2y$ & $\phantom+\sin(y)$\\
      2    & $\phantom+\cos(y)$
    \end{tabular}
  \end{center}
  \begin{align*}
    \int \frac{y^2 \sin(y)}{27}\,dy &= \frac{2y \sin(y) + \left( 2 - y^2 \right) \cos(y)}{27}\\
      &= \frac{6x \sin(3x) + \left( 2 - 9x^2 \right) \cos(3x)}{27}\\
    \int_{\frac\pi6}^{\frac\pi4} x^2 \sin(3x)\,dx &= \frac{9\sqrt2\pi^2 + 3\Cdot2^{7/2}\pi - 2^{11/2}}{864}
	- \frac{\pi}{27}\\
      &\approx 0.10007285555380578281.
  \end{align*}
\end{frame}

\begin{frame}[allowframebreaks]{$\displaystyle \int_{\frac\pi6}^{\frac\pi4} x^4 \cos(2x)\,dx$,頁}
  設 $y = 2x$。
  \[\int x^4 \cos(2x)\,dx = \int \frac{y^4 \cos(y)}{64}\,dy.\]
  \begin{center}
    \begin{tabular}{cc}
      $u$  & $v$\\
      \hline
      $y^4$  & $\phantom+\sin(y)$\\
      $4y^3$ & $\phantom+\cos(y)$\\
      $12y^2$& $        -\sin(y)$\\
      $24y$  & $        -\cos(y)$\\
      24     & $\phantom+\sin(y)$
    \end{tabular}
  \end{center}
  \begin{align*}
       & \int \frac{y^4 \cos(y)}{64}\,dy\\
    =\:& \frac{\left( y^4 - 12y^2 + 24 \right) \sin(y) + \left( 4y^3 - 24y \right) \cos(y)}{64}\\
    =\:& \frac{\left( 2x^4 - 6x^2 + 3 \right) \sin(2x) + \left( 4x^3 - 6x \right) \cos(2x)}{4}\\
       & \int_{\frac\pi6}^{\frac\pi4} x^4 \cos(2x)\,dx\\
    =\:& \frac{\pi^4 - 48\pi^2 + 384}{512}\\
    \phantom=\:& -\frac{\sqrt3\pi^4 + 12\pi^3 - 4\Cdot3^{7/2}\pi^2 - 648\pi + 8\Cdot3^{11/2}}{5184}\\
    \approx\:& 0.009975817894309118.
  \end{align*}
\end{frame}

\begin{frame}{指數函數乘以 sin 或 cos}
  \[\int \e^{cx} \trig(ax+b)\,dx\]
  設 $u = \trig(ax+b)$ 與 $v = \e^{cx}$,則原式變為
  \[\int \frac u c \,dv.\]
  \begin{theorem}
    \negskip
    \begin{align*}
      \int \e^{cx} \cos(ax+b)\,dx &= \frac{\e^{cx} \left( a\sin(ax+b) + c\cos(ax+b) \right)}{c^2 + a^2}\\
      \int \e^{cx} \sin(ax+b)\,dx &= \frac{\e^{cx} \left( c\sin(ax+b) - a\cos(ax+b) \right)}{c^2 + a^2}.
    \end{align*}
  \end{theorem}
\end{frame}

\begin{frame}{$\displaystyle \int \e^{cx} \cos(ax+b)\,dx$}
  \begin{solution}
    設 $\theta = ax+b$。
    \begin{align*}
      \int \e^{cx} \cos(\theta)\,dx &= \frac{\e^{cx} \cos(\theta)}{c} + \int \frac{a\e^{cx} \sin(\theta)}{c}\,dx\\
      \int \e^{cx} \sin(\theta)\,dx &= \frac{\e^{cx} \sin(\theta)}{c} - \int \frac{a\e^{cx} \cos(\theta)}{c}\,dx.
    \end{align*}
    \Negskip
    \begin{align*}
	 & \int \e^{cx} \cos(\theta)\,dx\\
      =\:& \frac{\e^{cx} \left( a\sin(\theta) + c\cos(\theta) \right)}{c^2}
	   -\int \frac{a^2 \e^{cx} \cos(\theta)}{c^2}\,dx\\
      =\:& \frac{\e^{cx} \left( a\sin(\theta) + c\cos(\theta) \right)}{c^2 + a^2}.
    \end{align*}
  \end{solution}
\end{frame}

\begin{frame}{$\displaystyle \int \e^{cx} \sin(ax+b)\,dx$}
  \begin{solution}
    設 $\theta = ax+b$。
    \begin{align*}
      \int \e^{cx} \sin(\theta)\,dx &= \frac{\e^{cx} \sin(\theta)}{c} - \int \frac{a\e^{cx} \cos(\theta)}{c}\,dx\\
      \int \e^{cx} \cos(\theta)\,dx &= \frac{\e^{cx} \cos(\theta)}{c} + \int \frac{a\e^{cx} \sin(\theta)}{c}\,dx.
    \end{align*}
    \Negskip
    \begin{align*}
	 & \int \e^{cx} \sin(\theta)\,dx\\
      =\:& \frac{\e^{cx} \left( c\sin(\theta) - a\cos(\theta) \right)}{c^2}
	   -\int \frac{a^2 \e^{cx} \sin(\theta)}{c^2}\,dx\\
      =\:& \frac{\e^{cx} \left( c\sin(\theta) - a\cos(\theta) \right)}{c^2 + a^2}.
    \end{align*}
  \end{solution}
\end{frame}

\begin{frame}{$\displaystyle \int \e^{3x} \sin(2x)\,dx$}
  \begin{solution}
    \negskip
    \begin{align*}
      \int \e^{3x} \sin(2x)\,dx &= \frac{\e^{3x} \sin(2x)}{3} - \int \frac{2\e^{3x} \cos(2x)}{3}\,dx\\
      \int \e^{3x} \cos(2x)\,dx &= \frac{\e^{3x} \cos(2x)}{3} + \int \frac{2\e^{3x} \sin(2x)}{3}\,dx.
    \end{align*}
    \begin{align*}
	 & \int \e^{3x} \sin(2x)\,dx\\
      =\:& \frac{\e^{3x} \left( 3 \sin(2x) - 2 \cos(2x) \right)}{9}
	   -\int \frac{4 \e^{3x} \sin(2x)}{9}\,dx\\
      =\:& \frac{\e^{3x} \left( 3 \sin(2x) - 2 \cos(2x) \right)}{13}.
    \end{align*}
  \end{solution}
\end{frame}

\begin{frame}{多項式乘以指數函數乘以 sin 或 cos}
  \begin{itemize}
    \item 基本上技巧同多項式乘以 sin 或 cos,也不適合用表格解法。
    \item 分部積分時,取多項式乘以 sin 或 cos 為 $u$。
  \end{itemize}
\end{frame}

\begin{frame}{$\displaystyle \int x \e^x \cos(x)\,dx$}
  \begin{solution}
    \Negskip
    \begin{align*}
      \int x \e^x \cos(x)\,dx &= x \e^x \cos(x) - \int \e^x (\cos(x) - x \sin(x))\,dx\\
      \int \e^x \cos(x)\,dx &= \frac{\e^x (\sin(x) + \cos(x))}{2}\\
      \int x \e^x \sin(x)\,dx &= x \e^x \sin(x) - \int \e^x (\sin(x) + x \cos(x))\,dx\\
      \int \e^x \sin(x)\,dx &= \frac{\e^x (\sin(x) - \cos(x))}{2}\\
      \int x \e^x \cos(x)\,dx &= \frac{\left( x-1 \right) \e^x \sin(x) + x \e^x \cos(x)}{2}.
    \end{align*}
  \end{solution}
\end{frame}

\begin{frame}
  \begin{center}
    \huge Tanks for your attention!
  \end{center}
\end{frame}
\end{CJK}
\end{document}
