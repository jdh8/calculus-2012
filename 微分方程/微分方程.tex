%!TEX encoding = UTF-8 Unicode
\documentclass{beamer}
\usepackage{amsmath,amsthm,amssymb}
\usepackage{CJKutf8}
\usepackage{graphicx}
\usepackage{hyperref}

\hypersetup{colorlinks,linkcolor=,unicode}
\useoutertheme{sidebar}
\usecolortheme{rose}
\usecolortheme{seahorse}
\newcommand{\Cdot} {\!\cdot\!}
\newcommand{\Left} {\mathopen{}\mathclose\bgroup\left}
\newcommand{\Right}{\aftergroup\egroup\right}
\newcommand  {\e}{\textup e}
\renewcommand{\i}{\textup i}
\newcommand  {\N}{\mathbb N}
\newcommand  {\Z}{\mathbb Z}
\newcommand  {\Q}{\mathbb Q}
\newcommand  {\R}{\mathbb R}
\renewcommand{\C}{\mathbb C}
\renewcommand{\today}{\number\year~年~\number\month~月~\number\day~日}
\newcommand{\Negskip}{\vskip -1em plus 2pt minus 2pt}
\newcommand{\negskip}{\vskip -2em plus 3pt minus 3pt}

\theoremstyle{remark}
  \newtheorem{remark}{Remark}

\title{微分方程}
\author[何震邦]{何震邦 \href{http://jdh8.org/}{\textless jdh8.org\textgreater}\\
    \href{http://creativecommons.org/licenses/by-sa/3.0/tw/deed.zh\textunderscore TW}{\includegraphics{by-sa.eps}}}

\begin{document}
\begin{CJK}{UTF8}{bsmi}
\maketitle

\section{微分方程}
\begin{frame}{常微分方程與偏微分方程}
  函數方程中,有待解函數的導函數項時,稱為微分方程。例如
  \begin{equation}
    \sin(x)\,x'' = \left( x + 1 \right) x' + x^2 \e^{-3x}. \label{ODE}
  \end{equation}
  \begin{equation}
    x^{(4)} + x''' + x'' + x = \cos(t) \label{LinearDE}
  \end{equation}
  以上微分方程只有常導數,所以是\textbf{常微分方程} (ODE).
  有偏導數的方程是\textbf{偏微分方程} (PDE),例如
  \begin{equation}
    u_{xxt} = u_y + 1. \label{PDE}
  \end{equation}
\end{frame}

\begin{frame}{階}
  微分方程的\textbf{階}是方程中最高階導數的階。例如 \eqref{ODE}
  是二階微分方程,\eqref{PDE} 是三階微分方程,而 \eqref{LinearDE}
  是四階微分方程。
\end{frame}

\begin{frame}{線性微分方程}
  \textbf{線性微分方程}可以寫成以下的形式:
  \[ \sum_{k=0}^n f_k(t) x^{(k)}(t) = g(t) \]
  其中 $t$ 是自變數,$x$ 是待解函數,$f_k$ 和 $g$ 是已知函數,如
  \eqref{LinearDE}。
\end{frame}

\begin{frame}{解}
  從足微分方程的函數稱為\textbf{解}或\textbf{特解}。解可能有自變數的範圍限制。
  \begin{example}
    在 $t > 0$ 時,$x = t^{-3/2}$ 是 $4t^2 x'' + 12t x' + 3x = 0$ 的解。
  \end{example}
  \begin{proof}
    \negskip
    \begin{align*}
      x'  &= -\frac32 t^{-5/2} \\
      x'' &= -\frac{15}{4} t^{-7/2}
    \end{align*}
    \[ 4t^2 \left( -\frac{15}{4} t^{-7/2} \right)
        + 12t \left( -\frac32 t^{-5/2} \right)
        + 3t^{-3/2} = 0. \qedhere \]
  \end{proof}
\end{frame}

\begin{frame}{初始條件}
  \textbf{初始條件}形如
  \[ x_k = x^{(k)}(t_0) \]
  例如
  \[ x_0 = x(t_0).\]
\end{frame}

\begin{frame}{初值問題}
  \textbf{初值問題}是有適當初始條件的微分方程。
  \begin{example}
    以下是初值問題。
    \[ t^2 x'' + 2t x' + 3x = 0 \]
    \begin{align*}
      x (4) &= 1 \\
      x'(4) &= 2.
    \end{align*}
  \end{example}
\end{frame}

\begin{frame}{有效範圍}
  對於有初始條件
  \[ x_k = x^{(k)}(t_0) \]
  的微分方程而言,\textbf{有效範圍}是包含 $t_0$ 且解有效的最大範圍。
\end{frame}

\begin{frame}{通解}
  \textbf{通解}是微分方程的解的最廣義的形式。
  \begin{example}
    \[ 2tx' + 4x = 3 \]
    的通解是
    \[ x = \frac{c}{t^2} + \frac34 \]
    其中 $c$ 是任意常數。
  \end{example}
\end{frame}

\begin{frame}{實際解}
  同時符合微分方程與附屬條件(像是初始條件)的解,是\textbf{實際解}。
  \begin{example}
    求解
    \[ 2tx' + 4x = 3 \qquad x(1) = -4.\]
    \begin{solution}
      已經有通解了,所以我們只要代入初始條件,求取待定常數 $c$。
      \[ 4 = \frac{c}{1^2} + \frac34 \qquad c = -\frac{19}{4} \]
      \[ x = \frac34 - \frac{19}{4t^2}. \]
    \end{solution}
  \end{example}
\end{frame}

\begin{frame}{顯式解與隱式解}
  若待解函數為 $x$ 而自變數為 $t$,\textbf{顯式解}形如
  \[ x = f(t).\]
  以其他方式寫出待解函數與自變數的關係的,為\textbf{隱式解},像是
  \[ F(x,t) = 0.\]
\end{frame}

\begin{frame}{從隱式解推出顯式解}
  \begin{example}
    \[ x' = \frac{t}{x} \qquad x(2) = -1 \]
    的隱式解為
    \[ x^2 = t^2 - 3 \]
    求顯式解。
    \begin{solution}
      乍看之下有 $x = \pm\sqrt{t^2 - 3}$ 兩解,但只有負的合初值條件。
      \[ x = -\sqrt{t^2 - 3}.\]
    \end{solution}
  \end{example}
\end{frame}

\section[一階 ODE]{一階常微分方程}
\begin{frame}{一階常微分方程}
  \textbf{一階常微分方程}為符合以下形式的方程式:
  \[ \frac{dx}{dt} = f(x,t) \]
  其中 $f$ 不含微分項。
\end{frame}

\subsection[線性]{一階線性微分方程}
\begin{frame}{方法 1:一階線性微分方程}
  \begin{theorem}
    給定微分方程
    \[f(t)\,x' + g(t)\,x = h(t).\]
    設 $A(t) = \dfrac{g(t)}{f(t)}$ 與 $B(t) = \dfrac{h(t)}{f(t)}$,將方程化為
    \[x' + Ax = B.\]
    則 $\e^{\int A\,dt}$ 為一積分因子,使得通解為
    \[x = \e^{-\int A\,dt} \left( c + \int \e^{\int A\,dt} B\,dt \right).\]
  \end{theorem}
\end{frame}

\begin{frame}{通解的證明}
  \begin{proof}
    設 $P = \int A\,dt$,則
    \begin{align*}
      x' + P'\,x &= B \\
      \e^P x' + \e^P P'\,x &= \e^P B \\
      \left( \e^P x \right)' &= \e^P B
    \end{align*}
    此時積分,引進積分常數 $c$。
    \[ \e^P x = c + \int \e^P B\,dt \]
    \begin{equation}
      x = \e^{-\int A\,dt} \left( c + \int \e^{\int A\,dt} B\,dt \right).
      \qedhere \label{LinearGeneralSol}
    \end{equation}
  \end{proof}
\end{frame}

\begin{frame}{有效區間}
  \begin{theorem}
    給定微分方程
    \[ x' + Ax = B \]
    與初始條件
    \[ x_0 = x(t_0).\]
    若函數 $A$, $B$ 在區間 $(a, b)$ 中連續,且 $a < t_0 < b$,則 $x$
    在此區間中有唯一解。
  \end{theorem}

  證明從略,因為解的形式可以通解 \eqref{LinearGeneralSol} 中獲得。
\end{frame}

\begin{frame}{特殊狀況}
  給定微分方程
  \[x' + Ax = B.\]
  若 $B$ 為零,即原方程齊次,則通解成為
  \[x = c\e^{-\int A\,dt}.\]
  若 $A$ 為常數,則 $\int A\,dt = At$,使通解成為
  \[x = \e^{-At} \left( c + \int \e^{At} B\,dt \right).\]
\end{frame}

\begin{frame}{常係數一階線性微分方程}
  給定微分方程
  \[x' + Ax = B.\]
  若 $A$ 與 $B$ 皆為常數,即原方程常係數,則設 $x^* = B/A$,
  \[x' + A \left( x - x^* \right) = 0\]
  再設 $y = x - x^*$,原方程簡化成常係數齊次方程
  $y' + Ay = 0$,通解顯然為
  \begin{align*}
    y &= c_1 \e^{-Ay} \\
    x &= c_2 \e^{-Ax} + x^*
  \end{align*}
  其中 $c_1$ 為積分常數,而 $c_2 = c_1 \e^{Ax^*}$.
\end{frame}

\begin{frame}{指數衰變}
  \begin{example}
    An exponential decay function $f(t) = y_0 e^{-kt}$ models the among of drug in the blood $t$ hours after an initial
    dose of $y_0 = 100$ mg is administered. Assume the half-life of a particular drug is 16 hours. How much time is
    required for the drug to reach 1\% of the initial dose (1 mg)?
    \begin{solution}
      \negskip
      \begin{align*}
	50 &= 100\e^{-16k}\\
	k  &= \frac{\ln(2)}{16}\\
	1  &= 100\e^{-\frac{\ln(2)}{16}t}
      \end{align*}
      \[t = 16\log_2(100) \approx 106.30169903639559513 \mbox{(小時)}.\]
    \end{solution}
  \end{example}
\end{frame}

\subsection[變數可分離]{變數可分離的微分方程}
\begin{frame}{方法 2:變數可分離的微分方程}
  \begin{theorem}
    給定微分方程
    \[A(x)\,B(t)\,x' = C(x)\,D(t).\]
    通解為
    \[\int \frac{A(x)}{C(x)}\,dx = \int \frac{D(t)}{B(t)}\,dt.\]
    證明從略。
  \end{theorem}
\end{frame}

\begin{frame}{有效區間}
  \begin{theorem}
    給定微分方程
    \[ P(x)\,x' = Q(t) \]
    與初始條件
    \[ x_0 = x(t_0) \]
    若在 $a < t < b$ 時,函數 $P$, $Q$ 處處連續,則在此有實際解
    \[ \int_{x_0}^x P(u)\,du = \int_{t_0}^t Q(v)\,dv.\]
  \end{theorem}

  由換元積分得證。
\end{frame}

\begin{frame}[allowframebreaks]{Logistic 成長,頁}
  \begin{example}
    A population grows according to the logistic differential equation $y' = 0.0003y \left( 2000-y \right)$. The initial
    population size is 800. Solve this differential equation and use the solution to predict to population size at time
    $t=3$.
  \end{example}
  本題為一個初期值問題
  \[\left\{ \begin{aligned}
      y'(t) &= 0.0003y(t) \left( 2000-y(t) \right)\\
      y(0)  &= 800
    \end{aligned} \right.\]
  原微分方程為變數可分離的微分方程。
  \[10000y' = 3y \left( 2000 - y \right)\]
  既然我們要求 $y(3)$,我們就從 $t=0$ 積到 $t=3$。
  \begin{align*}
    \int_{y(0)}^{y(3)} \frac{10000}{3y \left( 2000 - y \right)}\,dy &= \int_0^3 dt\\
    \int_{800}^{y(3)} \frac53 \left( \frac1y - \frac{1}{y-2000} \right) dy &= 3.
  \end{align*}
  此時被積函數有 0 和 2000 兩個不連續點,所以有效區間為 $(0, 2000)$。
  \[\ln(y(3)) - \ln(800) - \ln(2000 - y(3)) + \ln(1200) = \frac95\]
  \begin{align*}
    \ln \Left( \frac{y(3)}{2000-y(3)} \Right) &= \frac95 + \ln \Left( \frac23 \Right)\\
    \frac{y(3)}{2000-y(3)} &= \frac{2\e^{9/5}}{3}\\
    \frac{2000}{2000-y(3)} &= \frac{2\e^{9/5}}{3} + 1
  \end{align*}
  \[y(3) = \frac{4000\e^{9/5}}{2\e^{9/5}+3} \approx 1602.6304520653167546.\]
\end{frame}

\subsection[恰當--積分因子]{恰當微分方程}
\begin{frame}{方法 3:恰當微分方程}
  \begin{theorem}
    給定微分方程
    \[P(x,t)\,x' + Q(x,t) = 0\]
    其中
    \begin{equation*}
      \frac{\partial P}{\partial t} = \frac{\partial Q}{\partial x}.
    \end{equation*}
    通解為勢函數
    \[F(x,t) = \int Q\,dt + \int \left( P - \frac{\partial}{\partial x} \int Q\,dt \right) dx.\]
  \end{theorem}
\end{frame}

\begin{frame}{勢函數的存在}
  \begin{proof}
    設有函數 $F(x,t)$ 使得
    \[\frac{\partial F}{\partial x} = P \quad
      \mbox{且} \quad \frac{\partial F}{\partial t} = Q.\]
    此時若 $P$, $Q$, $P_t$ 與 $Q_x$
    在所討論的 $xt$-平面區域上連續,則在此區域中由\href
    {https://zh.wikipedia.org/wiki/\%E4\%BA\%8C\%E9\%98\%B6\%E5\%AF\%BC\%E6\%95\%B0\%E7\%9A\%84\%E5\%AF\%B9\%E7\%A7\%B0\%E6\%80\%A7}
    {二階導數的對稱性}得
    \[\frac{\partial P}{\partial t}
      = \frac\partial{\partial t} \frac{\partial F}{\partial x}
      = \frac\partial{\partial x} \frac{\partial F}{\partial t}
      = \frac{\partial Q}{\partial x}. \qedhere\]
  \end{proof}
\end{frame}

\begin{frame}{通解的證明}
  \begin{proof}
    設
    \[F = \int Q\,dt + \int \left( P - \frac{\partial}{\partial x} \int Q\,dt \right) dx.\]
    \[\frac{\partial F}{\partial x} = \frac{\partial}{\partial x} \int Q\,dt + P - \frac{\partial}{\partial x} \int Q\,dt
      = P.\]
    \begin{align*}
      \frac{\partial F}{\partial t}
	&= Q + \frac{\partial}{\partial t} \int \left( P - \frac{\partial}{\partial x} \int Q\,dt \right) dx\\
	&= Q + \int \left( \frac{\partial P}{\partial t} - \frac{\partial Q}{\partial x} \right) dx = Q. \qedhere
    \end{align*}
  \end{proof}
\end{frame}

\begin{frame}{積分因子}
  對於非恰當微分方程
  \[P(x,t)\,x' + Q(x,t) = 0\]
  我們可以試著尋找積分因子 $\mu(x,t)$ 使得
  \[\mu Px' + \mu Q = 0\]
  為恰當微分方程,即
  \begin{equation}
    \frac{\partial}{\partial t}\,\mu P = \frac{\partial}{\partial x}\,\mu Q \label{eq:Multiplier}
  \end{equation}
  其中 $\mu \ne 0$。
\end{frame}

\begin{frame}{積分因子為單變函數}
  \begin{theorem}
    給定非恰當微分方程
    \begin{equation}
      P(x,t)\,x' + Q(x,t) = 0. \label{eq:NonExact}
    \end{equation}
    若 $\dfrac{Q_x - P_t}{P} = h(t)$,則積分因子為
    \begin{equation}
      \mu(t) = \e^{\int h(t)\,dt}. \label{eq:xMultiplier}
    \end{equation}
    同理,若 $\dfrac{P_t - Q_x}{Q} = k(x)$,則積分因子為
    \[\mu(x) = \e^{\int k(x)\,dx}.\]
  \end{theorem}
\end{frame}

\begin{frame}{\eqref{eq:xMultiplier}式的證明}
  \begin{proof}
    設 $\mu(t)$ 為\eqref{eq:NonExact}的積分因子,則由\eqref{eq:Multiplier}得
    \[\mu\,\frac{\partial Q}{\partial x} - \mu\,\frac{\partial P}{\partial t} - P\,\frac{d\mu}{dt} = 0\]
    這是變數可分離的微分方程。我們著手解 $\mu$。
    \[\frac{\mu'}{\mu} = \frac{Q_x - P_t}{P} = h\]
    \negskip
    \begin{align*}
      \ln \mu &= \int h\,dt\\
      \mu &= \e^{\int h\,dt}. \qedhere
    \end{align*}
  \end{proof}
\end{frame}

\begin{frame}{$P$ 與 $Q$ 滿足柯西--黎曼方程時的積分因子}
  \begin{theorem}
    給定微分方程
    \begin{equation}
      P(x,t)\,x' + Q(x,t) = 0. \label{eq:CauchyRiemann}
    \end{equation}
    其中
    \[\frac{\partial Q}{\partial t} = \frac{\partial P}{\partial x} \quad \mbox{且} \quad \frac{\partial P}{\partial t} =
      -\frac{\partial Q}{\partial x}\]
    即 $P$ 與 $Q$ 滿足\href
    {http://zh.wikipedia.org/wiki/\%E6\%9F\%AF\%E8\%A5\%BF\%EF\%BC\%8D\%E9\%BB\%8E\%E6\%9B\%BC\%E6\%96\%B9\%E7\%A8\%8B}
    {柯西--黎曼方程},此時積分因子為
    \[\mu(x,t) = \frac{1}{P^2 + Q^2}.\]
  \end{theorem}
\end{frame}

\subsection[伯努力]{伯努力微分方程}
\begin{frame}{方法 4:伯努力微分方程}
  \begin{theorem}
    給定微分方程
    \[f(x)\,y' + g(x)\,y + h(x)\,y^n = 0\]
    其中 $n \ne 1$。設 $u(x) = y^{1-n}$,則原方程轉為線性的
    \[f(x)\,u' + \left( 1-n \right) g(x)\,u + \left( 1-n \right) h(x) = 0.\]
  \end{theorem}
\end{frame}

\subsection[齊次]{齊次微分方程}
\begin{frame}{齊次函數}
  \begin{definition}
    函數 $f(x,y)$ 稱為 $n$ 次齊次函數,等價於
    \begin{equation}
      f(tx,ty) = t^n f(x,y) \label{eq:Homogeneity}
    \end{equation}
    其中 $n$ 為常數。
  \end{definition}
  \begin{example}
    \begin{itemize}
      \item $f(x,y) = x^3 + 2x^2y + 3xy^2 + 4y^3$ 為 3 次齊次函數。
      \item $f(x,y) = x^2 + y^2 + 1$ 不是齊次函數。
    \end{itemize}
  \end{example}
\end{frame}

\begin{frame}{方法 5:齊次微分方程}
  \begin{theorem}
    給定微分方程
    \[P(x,t)\,x' = Q(x,t)\]
    其中 $P$ 與 $Q$ 皆為 $x$, $t$ 的齊次函數。設 $u(t) = x/t$,則原方程轉為變數可分離的方程。
  \end{theorem}
\end{frame}

\subsection[幾乎線性]{幾乎線性的微分方程}
\begin{frame}{方法 6:幾乎線性的微分方程}
  \begin{theorem}
    給定微分方程
    \[f(t)\,k'(x)\,x' + g(t)\,k(x) = h(t).\]
    設 $u(t) = k(x)$,則原方程轉為線性的
    \[f(t)\,u' + g(t)\,u = h(t).\]
  \end{theorem}
  \begin{remark}
    這方法在課本上罕見。
  \end{remark}
\end{frame}

\subsection[含線性分式]{含線性分式的微分方程}
\begin{frame}{方法 7:含線性分式的微分方程}
  \begin{theorem}
    給定微分方程
    \[x' = f \Left( \frac{a_1 x + b_1 t + c_1}{a_2 x + b_2 t + c_2} \Right).\]
    若 $a_1 x + b_1 t + c_1 = 0$ 與 $a_2 x + b_2 t + c_2 = 0$ 聯立方程有唯一解
    $(x,t) = (h,k)$,設 $X = x-h$ 與 $T = t-k$,則原方程轉為齊次的
    \[\frac{dX}{dT} = f \Left( \frac{a_1 X + b_1 T}{a_2 X + b_2 T} \Right).\]
  \end{theorem}
\end{frame}

\subsection[以 $t^n x$ 替代]{以 t\textasciicircum n x 替代}
\begin{frame}{方法 8:以 $t^n x$ 替代}
  \begin{theorem}
    給定微分方程
    \[x' + \frac{x f(t^n x)}{t} = 0\]
    其中 $n$ 為待定常數。設 $y = t^n x$,則原方程轉為變數可分離的
    \[\int \frac{1}{y \left( n - f(y) \right)}\,dy = \int \frac1t\,dt.\]
  \end{theorem}
\end{frame}

\begin{frame}{謝謝聆聽!}
  \begin{center}
    \includegraphics[width=0.3\textheight]{favicon}
  \end{center}
  \begin{itemize}
    \item \href{http://jdh8.org/}{部落格}
    \item \href{http://boards.jdh8.org/cal/}{討論版}
    \item \href{https://github.com/jdh8/calculus-2012}{系列教材}
  \end{itemize}
\end{frame}
\end{CJK}
\end{document}
