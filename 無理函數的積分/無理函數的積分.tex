%!TEX encoding = UTF-8 Unicode
\documentclass{beamer}
\usepackage{amsmath,amsthm,amssymb}
\usepackage{CJKutf8}
\usepackage{graphicx}
\usepackage{hyperref}

\hypersetup{colorlinks,linkcolor=,unicode}
\useoutertheme{sidebar}
\usecolortheme{rose}
\usecolortheme{seahorse}
\newcommand{\Cdot} {\!\cdot\!}
\newcommand{\Left} {\mathopen{}\mathclose\bgroup\left}
\newcommand{\Right}{\aftergroup\egroup\right}
\newcommand  {\e}{\textup e}
\renewcommand{\i}{\textup i}
\newcommand  {\N}{\mathbb N}
\newcommand  {\Z}{\mathbb Z}
\newcommand  {\Q}{\mathbb Q}
\newcommand  {\R}{\mathbb R}
\renewcommand{\C}{\mathbb C}
\newcommand{\sech}  {\operatorname{sech}}
\newcommand{\csch}  {\operatorname{csch}}
\newcommand{\arsinh}{\operatorname{arsinh}}
\newcommand{\op}  {\operatorname{op}}
\newcommand{\Elem}{\operatorname{Elem}}
\newcommand{\trig}{\operatorname{trig}}
\renewcommand{\today}{\number\year~年~\number\month~月~\number\day~日}
\newcommand{\Negskip}{\vskip -1em plus 2pt minus 2pt}
\newcommand{\negskip}{\vskip -2em plus 3pt minus 3pt}

\theoremstyle{remark}
  \newtheorem{remark}{Remark}

\title[積分無理函數]{無理函數的積分}
\author[何震邦]{何震邦 \href{mailto:jdh8@ms63.hinet.net}{\textless jdh8@ms63.hinet.net\textgreater}\\
    \href{http://creativecommons.org/licenses/by-sa/3.0/tw/deed.zh\textunderscore TW}{\includegraphics{by-sa.eps}}}

\begin{document}
\begin{CJK}{UTF8}{bsmi}
\maketitle

\begin{frame}{方法 5:反三角替代}
  在被積函數形如以下的無理函數時,進入本節的演算法。
  \begin{itemize}
    \item $f(x) \left( px + q \right)^m \left( a_1 x + b_1 \right)^{k_1} \left( a_2 x + b_2 \right)^{k_2}$
    \item $f(x) \left( px + q \right)^m \left( ax^2 + bx + c \right)^k$
  \end{itemize}
  其中 $f$ 是多項式,$m$ 是整數,$k$, $k_1$, $k_2$ 是奇數的一半,$a$, $a_1$, $a_2$, $b$, $b_1$, $b_2$, $c$, $p$, $q$
  是常數。
\end{frame}

\section{有理化}
\begin{frame}{$\displaystyle \int f(x) \left( px + q \right)^m \left( a_1 x + b_1 \right)^{k_1}
    \left( a_2 x + b_2 \right)^{k_2} dx$}
  其中 $f$ 是多項式,且 $k_1$ 與 $k_2$ 是奇數的一半。
  \begin{itemize}
    \item 若 $k_1 k_2 < 0$,設 $k_1 > 0$, $k_2 < 0$,不失一般性。若 $m \ge 0$,則 $g(x) = f(x) \left( px + q \right)^m$
      為多項式。
      \[g(x) \left( a_2 x + b_2 \right)^{k_2 - k_1} \left( a_1 x + b_1 \right)^{k_1} \left( a_2 x + b_2 \right)^{k_1}\]
      其中 $k_2 - k_1$ 為負整數。
    \item 否則設 $k_2 \ge k_1$,不失一般性。此時有理化為
      \[h(x) \left( px + q \right)^m \left( a_1 x + b_1 \right)^{k_1} \left( a_2 x + b_2 \right)^{k_1}\]
      其中 $h(x) = f(x)  \left( a_2 x + b_2 \right)^{k_2-k_1}$ 亦為多項式。
  \end{itemize}
\end{frame}

\begin{frame}{$\displaystyle \int f(x) \left( px + q \right)^m \left( ax^2 + bx + c \right)^k dx$}
  其中 $f$ 是多項式,且 $k$ 是奇數的一半。設 $y = px + q$,則原式化為
  \[g(y) \left( Ay^2 + By + C \right)^k\]
  其中 $g$ 是\href{http://zh.wikipedia.org/wiki/\%E6\%B4\%9B\%E6\%9C\%97\%E7\%BA\%A7\%E6\%95\%B0}{洛朗級數},$A$, $B$, $C$
  為常數。
\end{frame}

\section{多項式型}
\begin{frame}{多項式型}
  本節我們討論
  \[\int f(x) \left( ax^2 + bx + c \right)^k\]
  其中 $f$ 是多項式,且 $k$ 是奇數的一半。
\end{frame}

\subsection{基本型}
\begin{frame}{$\displaystyle \int \frac{1}{\sqrt{ax^2 + bx + c}}\,dx$}
  \begin{itemize}
    \item 若 $a > 0$ 且 $4ac > b^2$,則原式為
      \[\frac{\arsinh \Left( \frac{2ax + b}{\sqrt{4ac - b^2}} \Right)}{\sqrt a}.\]
    \item 若 $a > 0$ 且 $4ac < b^2$,則原式為
      \[\frac{\ln \Left| 2 \sqrt a \sqrt{ax^2 + bx + c} + 2ax + b \Right|}{\sqrt a}.\]
    \item 若 $a < 0$ 且 $4ac < b^2$,則原式為
      \[-\frac{\arcsin \Left( \frac{2ax + b}{\sqrt{b^2 - 4ac}} \Right)}{\sqrt{-a}}.\]
  \end{itemize}
\end{frame}

\begin{frame}{$\displaystyle \int \frac{1}{\sqrt{x - \alpha} \sqrt{x - \beta}}\,dx$}
  \begin{solution}
    設 $y = \sqrt{x - \alpha} + \sqrt{x - \beta}$。
    \[y' = \frac{1}{2 \sqrt{x - \alpha}} + \frac{1}{2 \sqrt{x - \beta}}
      = \frac{\sqrt{x - \alpha} + \sqrt{x - \beta}}{2 \sqrt{x - \alpha} \sqrt{x - \beta}}.\]
    \begin{align*}
      \int \frac{1}{\sqrt{x - \alpha} \sqrt{x - \beta}}\,dx &= \int \frac{2}{y}\,dy\\
	&= 2 \ln \Left| y \Right|\\
	&= 2 \ln \Left| \sqrt{x - \alpha} + \sqrt{x - \beta} \Right|\\
	&= \ln \Left| 2 \sqrt{x - \alpha} \sqrt{x - \beta} + 2x - \alpha - \beta \Right|.
    \end{align*}
  \end{solution}
\end{frame}

\begin{frame}{$\displaystyle \int \frac{1}{\sqrt{(x - \alpha) \left( x - \beta \right)}}\,dx$}
  \begin{solution}
    \begin{align*}
	 & \frac{d}{dx} \ln \Left| 2 \sqrt{(x - \alpha) \left( x - \beta \right)} + 2x - \alpha - \beta \Right|\\
      =\:& \frac{\dfrac{2x - \alpha - \beta}{\sqrt{(x - \alpha) \left( x - \beta \right)}} + 2}
	   {2 \sqrt{(x - \alpha) \left( x - \beta \right)} + 2x - \alpha - \beta}\\
      =\:& \frac{1}{\sqrt{(x - \alpha) \left( x - \beta \right)}}
    \end{align*}
  \end{solution}
\end{frame}

\begin{frame}{$\displaystyle \int \frac{1}{\sqrt{ax^2 + bx + c}}\,dx$,其中 $a > 0$ 且 $4ac < b^2$}
  設 $\alpha = \dfrac{-b - \sqrt{b^2 - 4ac}}{2a}$ 與 $\beta = \dfrac{-b + \sqrt{b^2 - 4ac}}{2a}$。
  \begin{solution}
  設 $y = ax$。
    \begin{align*}
      \int \frac{1}{\sqrt{ax^2 + bx + c}}\,dx &= \int \frac{\sqrt a}{\sqrt{(ax - a\alpha) \left( ax - a\beta \right)}}\,dx\\
	&= \int \frac{1}{\sqrt a \sqrt{(y - a\alpha) \left( y - a\beta \right)}}\,dy\\
	&= \frac{\ln \Left| 2 \sqrt a \sqrt{ax^2 + bx + c} + 2ax + b \Right|}{\sqrt a}.
    \end{align*}
  \end{solution}
\end{frame}

\begin{frame}{$\displaystyle \int_5^{15} \frac{1}{\sqrt{16x^2 - 1}}\,dx$}
  \begin{solution}
    當 $5 \le x \le 15$ 時,$\sqrt{16x^2 - 1} = \sqrt{4x + 1} \sqrt{4x - 1}$。
    \begin{align*}
      \int \frac{1}{\sqrt{16x^2 - 1}}\,dx &= \frac{\ln \Left| 8\sqrt{16x^2 - 1} + 32x \Right|}{4}\\
      \int_5^{15} \frac{1}{\sqrt{16x^2 - 1}}\,dx &= \frac{\ln \Left( 8\sqrt{3599} + 480 \Right)}{4}
	  - \frac{\ln \Left( 8\sqrt{399} + 160 \Right)}{4}\\
	&\approx 0.2747921059353482.
    \end{align*}
  \end{solution}
\end{frame}

\begin{frame}{$\displaystyle \int_5^{15} \frac{1}{\sqrt{16x^2 - 1}}\,dx$,另解}
  \begin{solution}
    當 $5 \le x \le 15$ 時,$\sqrt{16x^2 - 1} = \sqrt{4x + 1} \sqrt{4x - 1}$。
    \begin{align*}
      \int \frac{1}{\sqrt{16x^2 - 1}}\,dx &= \frac{2 \ln \Left| \sqrt{4x + 1} + \sqrt{4x - 1} \Right|}{4}\\
      \int_5^{15} \frac{1}{\sqrt{16x^2 - 1}}\,dx &= \frac{2 \ln \Left( \sqrt{61} + \sqrt{59} \Right)}{4}
	  - \frac{2 \ln \Left( \sqrt{21} + \sqrt{19} \Right)}{4}\\
	&\approx 0.2747921059353482.
    \end{align*}
  \end{solution}
  \begin{remark}
    這個解對人類比較友善,但對機器則否。因為它要多做兩次平方根。
  \end{remark}
\end{frame}

\begin{frame}{$\displaystyle \int \frac{1}{\sqrt{ax^2 + b}}\,dx$}
  \begin{itemize}
    \item 若 $a > 0$ 且 $b > 0$,則原式為
      \[\frac{\arsinh \Left( \frac{\sqrt a x}{\sqrt b} \Right)}{\sqrt a}.\]
    \item 若 $a > 0$ 且 $b < 0$,則原式為
      \[\frac{\ln \Left| 2 \sqrt a \sqrt{ax^2 + b} + 2ax \Right|}{\sqrt a}.\]
    \item 若 $a < 0$ 且 $b > 0$,則原式為
      \[\frac{\arcsin \Left( \frac{\sqrt{-a}x}{\sqrt b} \Right)}{\sqrt{-a}}.\]
  \end{itemize}
\end{frame}

\begin{frame}{$\displaystyle \int \frac{1}{\sqrt{ax^2 + b}}\,dx$,其中 $a > 0$ 且 $b > 0$}
  \begin{solution}
    設 $y = \arsinh \Left( \dfrac{\sqrt a x}{\sqrt b} \Right)$,則 $dx = \dfrac{\sqrt b \cosh(y)}{\sqrt a}\,dy$。
    \begin{align*}
      \int \frac{1}{\sqrt{ax^2 + b}}\,dx &= \int \frac{1}{\sqrt b \cosh y}\,dx\\
	&= \int \frac{1}{\sqrt a}\,dy\\
	&= \frac{\arsinh \Left( \frac{\sqrt a x}{\sqrt b} \Right)}{\sqrt a}.
    \end{align*}
  \end{solution}
\end{frame}

\begin{frame}{$\displaystyle \int \frac{1}{\sqrt{ax^2 + b}}\,dx$,其中 $a < 0$ 且 $b > 0$}
  \begin{solution}
    設 $y = \arcsin \Left( \dfrac{\sqrt{-a}x}{\sqrt b} \Right)$,則 $dx = \dfrac{\sqrt b \cos(y)}{\sqrt{-a}}\,dy$
    \begin{align*}
      \int \frac{1}{\sqrt{ax^2 + b}}\,dx &= \int \frac{1}{\sqrt b \cos(y)}\,dx\\
	&= \int \frac{1}{\sqrt{-a}}\,dy\\
	&= \frac{\arcsin \Left( \frac{\sqrt{-a}x}{\sqrt b} \Right)}{\sqrt{-a}}.
    \end{align*}
  \end{solution}
\end{frame}

\begin{frame}{$\displaystyle \int \frac{1}{\sqrt{ax^2 + bx + c}}\,dx$,其中 $a > 0$ 且 $4ac > b^2$}
  \begin{solution}
    設 $y = x + \dfrac{b}{2a}$。
    \begin{align*}
      \int \frac{1}{\sqrt{ax^2 + bx + c}}\,dx &= \int \frac{1}{\sqrt{ay + \frac{4ac - b^2}{4a}}}\,dy\\
	&= \frac{\arsinh \Left( \frac{2ay}{\sqrt{4ac - b^2}} \Right)}{\sqrt a}\\
	&= \frac{\arsinh \Left( \frac{2ax + b}{\sqrt{4ac - b^2}} \Right)}{\sqrt a}.
    \end{align*}
  \end{solution}
\end{frame}

\begin{frame}{$\displaystyle \int \frac{1}{\sqrt{ax^2 + bx + c}}\,dx$,其中 $a < 0$ 且 $4ac < b^2$}
  \begin{solution}
    設 $y = x + \dfrac{b}{2a}$。
    \begin{align*}
      \int \frac{1}{\sqrt{ax^2 + bx + c}}\,dx &= \int \frac{1}{\sqrt{ay - \frac{b^2 - 4ac}{4a}}}\,dy\\
	&= \frac{\arcsin \Left( \frac{-2ay}{\sqrt{b^2 - 4ac}} \Right)}{\sqrt{-a}}\\
	&= -\frac{\arcsin \Left( \frac{2ax + b}{\sqrt{b^2 - 4ac}} \Right)}{\sqrt{-a}}.
    \end{align*}
  \end{solution}
\end{frame}

\subsection{一般型}
\begin{frame}{$\displaystyle \int \frac{px + q}{\sqrt{ax^2 + bx + c}}\,dx$}
  \begin{solution}
    \begin{align*}
	 & \int \frac{px + q}{\sqrt{ax^2 + bx + c}}\,dx\\
      =\:& \frac{p}{2a} \int \frac{2ax + b}{\sqrt{ax^2 + bx + c}}\,dx
	   +\int \frac{q - \frac{bp}{2a}}{\sqrt{ax^2 + bx + c}}\,dx\\
      =\:& \frac{p \sqrt{ax^2 + bx + c}}{a} + \int \frac{2aq - bp}{2a \sqrt{ax^2 + bx + c}}\,dx.
    \end{align*}
  \end{solution}
\end{frame}

\begin{frame}{$\displaystyle \int \left( ax^2 + bx + c \right)^{n + \frac12} dx$,其中 $n \in \N_0$}
  \begin{solution}
    設 $R = ax^2 + bx + c$。
    \begin{align*}
	 & \int R^{n + \frac12} dx\\
      =\:& x R^{n + \frac12}
	   -\int \frac{\left( 2n + 1 \right) \left( 2ax^2 + bx \right) R^{n - \frac12}}{2}\,dx\\
      =\:& \frac{x R^{n + \frac12}}{2n + 2}
	   +\int \frac{\left( 2n + 1 \right) \left( bx + c \right) R^{n - \frac12}}{4n + 4}\,dx\\
      =\:& \frac{\left( 2ax + b \right) R^{n + \frac12}}{\left( 4n+4 \right) a}
	   +\int \frac{\left( 2n + 1 \right) \left( 4ac - b^2 \right) R^{n - \frac12}}{\left( 8n+8 \right) a}\,dx
    \end{align*}
  \end{solution}
\end{frame}

\begin{frame}{$\displaystyle \int \frac{x^2}{\sqrt{x^2 + x + 1}}\,dx$}
  \begin{solution}
    \begin{align*}
      \int \frac{x^2}{\sqrt{x^2 + x + 1}}\,dx &= \int \sqrt{x^2 + x + 1}\,dx - \int \frac{x + 1}{\sqrt{x^2 + x + 1}}\,dx\\
	&= \frac{2x + 1}{4 \sqrt{x^2 + x + 1}} - \int \frac{8x + 5}{8 \sqrt{x^2 + x + 1}}\,dx\\
	&= \frac{2x - 3}{4 \sqrt{x^2 + x + 1}} - \int \frac{1}{8 \sqrt{x^2 + x + 1}}\,dx\\
	&= \frac{2x - 3}{4 \sqrt{x^2 + x + 1}} - \int \frac{\arsinh \Left( \frac{2x + 1}{\sqrt3} \Right)}{8}\,dx
    \end{align*}
  \end{solution}
\end{frame}

\subsection[Hermite red.]{Hermite reduction}
\begin{frame}{Hermite reduction}
  設 $n$ 為正整數。若多項式 $R$ 與 $R'$ 互質。根據多項式的\href
  {http://zh.wikipedia.org/wiki/\%E8\%B2\%9D\%E7\%A5\%96\%E7\%AD\%89\%E5\%BC\%8F}{貝祖等式},我們能以擴展的輾轉相除法求兩多項式
  $B$, $C$ 使得
  \[\frac{2A}{1-2n} = BR' + CR\]
  其中 $\deg(B) < \deg(R)$。因此
  \begin{align*}
    \frac{A}{R^{n + \frac12}} &= \frac{\left( 1-2n \right) BR'}{2R^{n + \frac12}}
	 +\frac{\left( 1-2n \right) C}{2R^{n - \frac12}}\\
      &= \frac{B'}{R^{n - \frac12}} - \frac{\left( 2n-1 \right) BR'}{2R^{n + \frac12}}
	 -\frac{B' + \left( 2n-1 \right) C}{2R^{n - \frac12}}\\
    \int \frac{A}{R^{n + \frac12}} &= \frac{B}{R^{n - \frac12}} - \int \frac{B' + \left( 2n-1 \right) C}{2R^{n - \frac12}}.
  \end{align*}
\end{frame}

\begin{frame}{$\displaystyle \int \frac{px + q}{\left( ax^2 + bx + c \right)^{3/2}}\,dx$}
  設 $A = px + q$ 與 $R = ax^2 + bx + c$,其中 $4ac \ne b^2$。
  \begin{align*}
    4aR &= \left( 2ax + b \right) R' + 4ac - b^2\\
    -2A &= -\frac{pR'}{a} + \frac{bp - 2aq}{a}\\
	&= -\frac{pR'}{a} + \frac{\left( 2aq - bp \right) \left( \left( 2ax + b \right) R' - 4aR \right)}
	   {a \left( 4ac - b^2 \right)}.
  \end{align*}
  設 $B = \dfrac{\left( 2aq - bp \right) \left( 2ax + b \right)}{a \left( 4ac - b^2 \right)} - \dfrac p a$ 與
  $C = -\dfrac{4 \left( 2aq - bp \right) R}{4ac - b^2}$。
  \begin{align*}
    -2A &= BR' + CR\\
    \int \frac{A}{R^{3/2}} &= \frac{B}{\sqrt R} - \int \frac{B' + C/2}{\sqrt R} = \frac{B}{\sqrt R}.
  \end{align*}
\end{frame}

\begin{frame}{$\displaystyle \int \frac{4x + 5}{\left( x^2 + 2x + 3 \right)^{3/2}}\,dx$}
  \begin{solution}
    設 $A = 4x + 5$ 與 $R = x^2 + 2x + 3$。
    \begin{align*}
       2R &= \left( x + 1 \right) R' + 4\\
      -2A &= -4R' - 2 = \frac{\left( x - 7 \right) R'}{2} - R.
    \end{align*}
    \[\int \frac{4x + 5}{R^{3/2}}\,dx = \frac{x - 7}{2 \sqrt{x^2 + 2x + 3}}.\]
  \end{solution}
\end{frame}

\section{洛朗級數型}
\begin{frame}{$\displaystyle \int \frac{1}{x^m \sqrt{ax^2 + bx + c}}\,dx$}
  \begin{solution}
    設 $y = 1/x$。
    \begin{align*}
      \int \frac{1}{x^m \sqrt{ax^2 + bx + c}}\,dx &= -\int \frac{y^{m-2}}{\sqrt{\frac b y + \frac{a}{y^2} + c}}\,dy\\
	&= -\int \frac{y^{m-2} \left| y \right|}{\sqrt{cy^2 + by + a}}\,dy.
    \end{align*}
  \end{solution}
  \begin{remark}
    \[\frac{x}{|x|} = \begin{cases}1 & \mbox{ 若 }x > 0\\ -1 & \mbox{ 若 }x < 0.\end{cases}\]
  \end{remark}
\end{frame}

\subsection{基本型}
\begin{frame}{$\displaystyle \int \frac{1}{x \sqrt{ax^2 + bx + c}}\,dx$}
  \begin{itemize}
    \item 若 $c > 0$ 且 $4ac > b^2$,則原式為
      \[-\frac{\arsinh \Left( \frac{bx + 2c}{\sqrt{4ac - b^2} |x|} \Right)}{\sqrt c}.\]
    \item 若 $c > 0$ 且 $4ac < b^2$,則原式為
      \[-\frac{\ln \Left| \frac{2 \sqrt c \sqrt{ax^2 + bx + c}}{|x|} + \frac{2c}{|x|} + b \Right|}{\sqrt c}.\]
    \item 若 $c < 0$ 且 $4ac < b^2$,則原式為
      \[\frac{\arcsin \Left( \frac{bx + 2c}{\sqrt{b^2 - 4ac} |x|} \Right)}{\sqrt{-c}}.\]
  \end{itemize}
\end{frame}

\begin{frame}{$\displaystyle \int \frac{1}{x \sqrt{ax^2 + bx + c}}\,dx$,若 $c > 0$ 且 $4ac > b^2$}
  \begin{solution}
    設 $y = 1/x$。
    \begin{align*}
      \int \frac{1}{x \sqrt{ax^2 + bx + c}}\,dx &= -\int \frac{|y|}{y \sqrt{cy^2 + by + a}}\,dy\\
	&= -\frac{\arsinh \Left( \frac{\left( 2cy + b \right) |y|}{y \sqrt{4ac - b^2}} \Right)}{\sqrt c}\\
	&= -\frac{\arsinh \Left( \frac{bx + 2c}{\sqrt{4ac - b^2} |x|} \Right)}{\sqrt c}.
    \end{align*}
  \end{solution}
\end{frame}

\begin{frame}{$\displaystyle \int \frac{1}{x \sqrt{ax^2 + bx + c}}\,dx$,若 $c > 0$ 且 $4ac < b^2$}
  \begin{solution}
    設 $y = 1/x$。
    \begin{align*}
      \int \frac{1}{x \sqrt{ax^2 + bx + c}}\,dx &= -\int \frac{|y|}{y \sqrt{cy^2 + by + a}}\,dy\\
	&= -\frac{\ln \Left| 2 \sqrt c \sqrt{cy^2 + by + a} + 2c|y| + b \Right|}{\sqrt c}\\
	&= -\frac{\ln \Left| \frac{2 \sqrt c \sqrt{ax^2 + bx + c}}{|x|} + \frac{2c}{|x|} + b \Right|}{\sqrt c}.
    \end{align*}
  \end{solution}
\end{frame}

\begin{frame}{$\displaystyle \int \frac{1}{x \sqrt{ax^2 + bx + c}}\,dx$,若 $c < 0$ 且 $4ac < b^2$}
  \begin{solution}
    設 $y = 1/x$。
    \begin{align*}
      \int \frac{1}{x \sqrt{ax^2 + bx + c}}\,dx &= -\int \frac{|y|}{y \sqrt{cy^2 + by + a}}\,dy\\
	&= \frac{\arcsin \Left( \frac{\left( 2cy + b \right) |y|}{y \sqrt{b^2 - 4ac}} \Right)}{\sqrt{-c}}\\
	&= \frac{\arcsin \Left( \frac{bx + 2c}{\sqrt{b^2 - 4ac} |x|} \Right)}{\sqrt{-c}}.
    \end{align*}
  \end{solution}
\end{frame}

\begin{frame}{$\displaystyle \int \frac{1}{x \sqrt{ax^2 + b}}\,dx$}
  \begin{itemize}
    \item 若 $b > 0$ 且 $a > 0$,則原式為
      \[-\frac{\arsinh \Left( \frac{\sqrt b}{\sqrt a |x|} \Right)}{\sqrt b}.\]
    \item 若 $b > 0$ 且 $a < 0$,則原式為
      \[-\frac{\ln \Left| \frac{2 \sqrt b \sqrt{ax^2 + b}}{|x|} + \frac{2b}{|x|} \Right|}{\sqrt b}.\]
    \item 若 $b < 0$ 且 $a > 0$,則原式為
      \[-\frac{\arcsin \Left( \frac{\sqrt{-b}}{\sqrt a |x|} \Right)}{\sqrt{-b}}.\]
  \end{itemize}
\end{frame}

\subsection{一般型}
\begin{frame}{$\displaystyle \int \frac{1}{x^2 \sqrt{ax^2 + bx + c}}\,dx$}
  \begin{solution}
    設 $y = 1/x$。
    \begin{align*}
	 & \int \frac{1}{x^2 \sqrt{ax^2 + bx + c}}\,dx\\
      =\:& -\int \frac{|y|}{\sqrt{cy^2 + by + a}}\,dy\\
      =\:& -\frac{y \sqrt{cy^2 + by + a}}{c \left| y \right|} + \int \frac{b \left| y \right|}
	   {2c^{3/2}y \sqrt{cy^2 + by + a}}\,dy\\
      =\:& -\frac{\sqrt{ax^2 + bx + c}}{cx} + \int \frac{b \left| y \right|} {2c^{3/2}y \sqrt{cy^2 + by + a}}\,dy.
    \end{align*}
  \end{solution}
\end{frame}

\begin{frame}{$\displaystyle \int \frac{1}{x^3 \sqrt{ax^2 + bx + c}}\,dx$}
  \begin{solution}
    設 $y = 1/x$。
    \small
    \begin{align*}
	 & \int \frac{1}{x^3 \sqrt{ax^2 + bx + c}}\,dx = -\int \frac{y \left| y \right|}{\sqrt{cy^2 + by + a}}\,dy\\
      =\:& -\int \frac{y \sqrt{cy^2 + by + a}}{c \left| y \right|}\,dy + \int \frac{\left( by + a \right) |y|}
	   {cy \sqrt{cy^2 + by + a}}\,dy\\
      =\:& -\frac{\left( 2cy + b \right) \sqrt{cy^2 + by + a}}{4c^2} + \int \frac{\left( 8bcy + 4ac + b^2 \right) |y|}
	   {8c^2 y \sqrt{cy^2 + by + a}}\,dy\\
      =\:& \frac{\left( 3b - 2cy \right) \sqrt{cy^2 + by + a}}{4c^2} + \int \frac{\left( 4ac - 3b^2 \right) |y|}
	   {8c^2 y \sqrt{cy^2 + by + a}}\,dy\\
      =\:& \left( \frac{3b}{4c^2 x} - \frac{1}{2cx^2} \right) \sqrt{ax^2 + bx + c} + \int \frac{\left( 4ac - 3b^2 \right) |y|}
	   {8c^2 y \sqrt{cy^2 + by + a}}\,dy.
    \end{align*}
  \end{solution}
\end{frame}

\begin{frame}
  \begin{center}
    \huge Thanks for your attention!
  \end{center}
\end{frame}
\end{CJK}
\end{document}
